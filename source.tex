\documentclass[12pt]{article}
% support both greek and english
\usepackage[utf8]{inputenc}
\usepackage[greek,english]{babel}
\usepackage{alphabeta}

\usepackage{amsmath}
\usepackage[colorlinks = true, urlcolor = blue]{hyperref}
\usepackage{graphicx}
\usepackage{indentfirst}

\setlength{\parindent}{1em}
\setlength{\parskip}{1em}

\title{PCB refactoring}
\author{The usual suspect}
\date{}

\begin{document}

\maketitle
\tableofcontents{}
\pagebreak

\section{Introduction}

Something crucial that I understood about reliability, academia and research, citation, resources and so on is that it would be wise to use only papers and books as a reference and not blog posts. Basically as I can recall in the two reports that I wrote about PCB and thermal properties, I have a lot of information based on blog posts. It is necessarily bad, but it would be better to use papers and books. In books you can find more general and informal info. In papers more delicated things like EMI, signal integrity and on. So we need to adjust to that!

\section{Identify the design}

Signals rise time. I2C, clock, SPI and so on. Logic levels, compare with wavelength and introduce the sine wave analysis when dealing with sine waves. Transmission line theory or not? But most of the cases it would be wise to take into account transmission line theory considerations, high speed design guidelines. The question is, is circuit theory enough for my board to be functional and to pass the tests? Rules of thumbs are ...

What are the differences between digital design, analog design, high speed design, rf design, microwave design, high power design

\subsection{Sine waves}

In literature and most of the technical articles, sine waves are the input signals for the transmission line theory. But what will happen when we are dealing with square waves. The answer is fourier transform. We can think the digital wave as a group of sines with different frequencies. Knee frequency, bandwidth, filters and so on! Rule of thumb for integrating tranmission line considerations in the design, half wavelength and rise time.

Resources for this are the Daniel Beerek, altium live, referred to digital spectrum. The book that you based the handbook for the PCB, eric bogatin EMI from the first time and some articles, I guess. We will see

\section{High speed design}

We need to mention Signal integrity, power integrity, EMI

If we decide to take into account transmission line theory then we need to address the following:
\begin{itemize}
		\item Reflection
		\item Impedance matching
		\item Radiation, interference, coupling
		\item Timing. What about tuning?
		\item skin effect. But this happen I think too extreme frequencies. The current is located to the surface of the conductor and it isnt uniform distributed across the conductor
		\item About impedance matching, dont use copper thickness as an option to adjust impedance due to current requirements. Play with width and length is better, Stubs are used for impedance matching.
\end{itemize}

Most of them are related with each others and the basic principle is that it is all about the space. The energy is in the fields and in order for the energy to propagate to the load  a pair of conductors and a space between of them is necessary. The field is in the space and not in the conductors for the energy propagation. Having that in mind, a PCB designer now becomes a space designer. The geometry of the conductors should be configured in such way, to guide the desired energy from the driver to the desired load. So in order to do that an unbroken dielectric and an unbroken return path are necessary for this guidance. We guide fields, we guide energy, we design, configure space when dealing with high speed design. 

Connect circuit theory with electromagnetism. Transmission line theory! Its all about geometry of the conductors and space!

\subsection{Tips and trick}
\begin{itemize}
		\item Place the return path as close as possible and further the distance between different signals. Mention the narrow path and the image with the fields. Microstrip vs stripline, EMI and so on maybe.
		\item ground transition vias
		\item unbroken return path and unbroken dielectric
		\item Make it short
		\item About the width should be suited according the impedance matching criteria and plus to be short. Most of the time 50 ohm impedance. Dont forget the vias too. But in general in order to minimize the resistance, the thicker and wider the better, more copper, better heat dissipation and less resistance.
		\item What about the inputs and the output, such as cables, headers and so on?
		\item Shielding, vias stiching
		\item Give room to breathe, aka routing space. Further the distance between ICs and dont pack everything in small areas. Configure first your mechanical requirements, place the pins and so on. \href{https://www.autodesk.com/products/eagle/blog/top-10-pcb-component-placement-tips-pcb-beginner/}{source}
		\item unrelated signals should propably travel prependicular. Remember bottom trace should be vertical with the signal layer above. To avoid coupling in 2 layer stack up. If you have a reference plane then you dont really need to consider it.
\end{itemize}

\section{Materials}

Just an informative blog by Autodesk, \href{https://www.autodesk.com/products/eagle/blog/top-10-tips-high-speed-pcb-design/}{this}

\section{No transmission line no problems}

If you dont have transmission lines then you will still have a lot of stuff to worry about. Radiation is still a thing and so on. For example dont think that equipotential surfaces dont have current. You may have no votlage drop or phase during an interconnection but the current created across the equipotential, yes you can have zero voltage drop and current, will radiate. The fact is that transmissio line, the reflection can cause many problems and also in general the emissions are greater because we are reaching antenna dimensions.

Επίσης όσον αφορά την επιστροφή του ρεύματος με τους πυκνωτές και τα πηνία και το coupling, αυτά ισχύουν και στα μικρά interconnection, διοτί μικρό interconnection απλά σημαίνει ότι δεν έχεις voltage distribution across the interconnection, αλλά το γεγονός οτι η τάση αλλάζει άρα αφου ο πυκνωτής και τα πηνία βλέπουν τάση και ρεύμα που αλλάζει δρουν αναλόγως, απλά οι πυκνωτές καθόλη τη διαδρομή της γραμμής θα βλέπουν την ίδια τάση και δεν θα έχουμε κυματική συμπεριφορά, άρα μάλλον το distributed model και reflection είναι γενικά δύο διαφορετικά πράγματα. Βασικά ντάξει το κονσεπτ του distributed είναι ότι ο χώρος επηρεάζει, οπότε οκευ, αλλά radiation is still radiation. Συνοψίζοντας the above rule of thumb wont solve all the problems, you just dont worry much about reflection and such.

When unterminated transmission lines are no problem, when the rule of thumb above is applied. Eric bogati page 595, is good, refer to time dealy which is practically the same with the above rule of thumbs. Good! So you still gonna have problems even if short interconnection, crosstalk and so on. But no reflection, maybe.

This is very interesting. Transmission line is a cap. \href{https://www.polarinstruments.com/support/cits/Critical_length.pdf}{source}

\section{Assembly}

After the design and the bare board fabrication have finished, the time for soldering\footnote{Soldering is used both to attach components physically to the PCB and to provide electrical conductivity between the component’s leads and the PCB traces} and component placement has come (populated board). PCBs can be assembled \textbf{manually} or by \textbf{automatic} machinery. Manually assembled is related with placing and soldering the components by hand. This is suitable for prototypes and low volume production. A mixed process with hand placement and automatic soldering can also be used as a strategy regarding the manual method. For the automated assembly, pick and place machines are substituting the hands of the assemblers and the soldering process can be divided to reflow and wave soldering. 

% source the orcad book

The first one is used for SMDs and the second one mostly for through hole or for a mixed set up with both package types. 
As the high speed applications are increasing, SMT is replacing the through-hole due to better electrical performance (inductance and capacitance of the leads) and size costraints. Thus the wave soldering is used less except some special applications (e.g. power devices).
% \href{https://en.wikipedia.org/wiki/Wave_soldering}{source}.

%If wave soldering is going to be used then additional rules should be integrated in the design phase for a successful soldering (see DFM bullet for wave soldering)



%\href{https://uk.beta-layout.com/pcb/technology/assembly_guide.html}{source}, hitchhiker guide and %\href{https://www.seeedstudio.com/blog/2019/06/12/how-to-generate-assembly-files-and-why-they-are-important/}{this}



\section{Workflow: From design to reality} 

Now that we developed an intuition about the PCB, we will try to make it more specific and analyze the design aspect. What engineers actually do before the PCB is sent for fabrication?

The development cycle of the PCB could be divided into six main parts: 1) component selection, requirements and specifications, 2) design (PCB layout), 2) simulation, 3) manufacturing, 4) assembly and 5) testing. Testing is integrated both in the assembly and manufacturing.
%However, as we will see, the PCB engineers can do the functional testing themselves without the assembly house. 
The design can be divided to design for manufacturing, for testing and for performance.  

For the design part, specialized software tools are used in the industry. These are called CAD or EDA and provide features such as schematic capture, library management, PCB layout and generation of standardized files for fabrication along with other of course capabilities to make the life of designers easier. The engineer, very briefly, first organize the libraries for the components, then captures the schematic and finally starts the PCB layout, placing the components and routing the connections. Actually in large corporations, a lot of people are contributing to this: library manager, circuit engineer, layout engineer, test engineer.

A free and open source tool used by the AcubeSAT team for the first engineering models of OBDH and EPS is the \textbf{KiCad}. To start learning KiCad, useful resources are the well-written documentation from \href{http://docs.kicad-pcb.org/}{KiCad website} and a detailed book, KiCad like a pro. Of course, the best way to learn a tool is start using it!
%We will emphasize in the design process about the electrical performance of the PCB and we will also write down some important notes for design for testing and design for manufacturing.

\section{Libraries}

Having reliable and well configured libraries is a very important part of the PCB making. They are the fundamental blocks of the design, the bricks of a building. Always check if the data is compliant with the needs of the fabricators and the datasheets, especially for the pinout and the dimensions.

The libraries is consisted of the schematic symbols used for the schematic capture, the footprints for the PCB layout and the 3D models.

\subsection{Schematic symbols}

\subsection{Footprints}
%When creating a footprint, you should verify that the solder fillets meet the IPC criteria, and the footprint meets the DFM guidelines from your vendor.\href{https://www.nuvation.com/resources/article/pcb-layout-creating-perfect-smt-footprints}{source}.

%Most of the footrpints in the current designes having as source the samacsys library, \href{https://www.samacsys.com/pcb-library-standards/}{link}. The footrpints provided according to Samacys are compliant with IPC-7351B.

%Are Kicad footrpints in the default libraries IPC compliant?\href{https://www.google.com/url?sa=t&rct=j&q=&esrc=s&source=web&cd=&cad=rja&uact=8&ved=2ahUKEwj6w7y4zffqAhXJXhUIHVa9BUkQFjAAegQIBhAB&url=https%3A%2F%2Fforum.kicad.info%2Ft%2Fwhy-are-the-kicad-library-conventions-non-ipc-compliant%2F3678&usg=AOvVaw0usF0MtB9ulD66RZoquX_r}{source}

%Should ask yourself if standard compliance meet the vendor requirements for manufacturability? Or should desgin footrpints checking the documentation of the your vendors? Dont confuse yourself with the routing manufactruing requreiments with the footrpint requirements. In most cases the standard compliance has taken into account the general anufacturing requirements. But we need to investigate more about the footrpint criteria...

%What are footprints? 
Footprint or land pattern is an arrangement of exposed copper areas (pads) for the physical attachment and electric connection of the components with the bare board. In other words it is the interface between the board and the component. Manufacturers tend to work with standardized footprints to ensure consistency, quality and productivity. The most well know industrial standard for this purpose is the IPC-7351. 

Should the designers build all the footprints from the scratch for each component? Usually component vendors (TI, ST Microelectronics, NXP) provide the required data (schematic symbols, footrpints, 3D models) compliant with IPC for a wide range of EDA tools. Otherwise designers can make use of other third party libraries like SnapEDA, Ultralibrarian, SamacSys. In the engnineering model of the OBDH, most of the footprints mined from the SamacSys library (IPC-7351B). There is also a quiet useful \href{https://www.samacsys.com/kicad-libraries/}{plugin} for combining Kicad and SamacSys.

If you can't find or build one by IPC wizards (some EDA tools like Altium feature an integrated wizard to build IPC footprints), then most of the times, especially for the custom packages, the vendors will provide in the documentation guidelines on how to configure it. For example in case of the ADCS subsystem, the RM3100 is a custom package developed by PNI and in the figure 1 is depicted the recommended data for the footprint's implementation, \href{https://www.pnicorp.com/wp-content/uploads/RM3100-Sensor-Suite-User-Manual-R07-1-2.pdf}{source}, \href{https://web.archive.org/web/20200812135747/https://www.pnicorp.com/wp-content/uploads/RM3100-Sensor-Suite-User-Manual-R07-1-2.pdf}{archived}. When a package isn't standardized, it will probably increase the cost, because the manufacturer should adapt their flow to meet the needs of this particular component. %\href{https://resources.altium.com/p/working-ipc-compliant-footprint-models}{source}.

\begin{figure}
	\centering
	\includegraphics[keepaspectratio, width = \textwidth]{assets/rm3100_foot.png}
	\caption{RM3100 recommended custom footprint configuration}
\end{figure}



\textbf{How important are footrpints?}
A lot of problems originates from poor library implementation. So, it would be wise to always review your libraries. Precisely, check each pad if it is properly assigned for each type of layer (e.g. copper, solder paste, soldermask). Sometimes ready footrpints offered by third party libraries might have some issues like assigning copper areas as a drawing (no electric connection) that can cause many problems. This actually happened in the EPS's board. The drawing had the exact same color with the layer associated with the copper one, so it was very error-prone! For the mistakes made in the engineering models you can check the AcubeSAT-OBC-EC-012.

%If you want to learn more about footprints, there is a nice extensive \href{https://www.youtube.com/watch?v=cMxXea16Hxc}{youtube video} by Robert Fenerec.

\subsection{3D models}

3D models help a lot with the visualization and can be used for debugging. For instance, by viewing your board in 3D, you can inspect if objects are overlapping and if everything fits together. This of course can be done in the PCB layout enabling the layer that is dedicated for the mechanical dimensions (if your footprint contain the required data) like the Courtyard in KiCad, as we have mentioned previously.

\section{Fabrication data}

\subsection{Manufacturing data}

Next step after the finished design is the bare board fabrication. For this purpose, the generation of files with special format called \textbf{Gerber} is the de facto standardized way to connect manufacturers and designers. Each EDA tool has the option to generate these type of files. In short they contain information about the copper of each layer of the PCB, the soldermask and the silkscreen. The \textbf{NC} (Numeric Controlled) \textbf{drill files} are also generated along with the Gerber ones to specify the data for the drilling machines to create the holes of the PCB.

% source, easy googling
%Any tips and trick for the Gerbers? XY coordinates and origin?

\subsection{Assembly data}

Each assembly house requires certain formats about the data requested for the process, so some differences might exist among them. Assembly data is typically referred to the following:

\begin{itemize}	
	\item Bill of Materials (\textbf{BOM}). The materials used for a cooking recipe! It is usually a csv file that lists everything that the assembler needs to know about the reference designators (component identifiers) and the manufacturing part numbers of the ordered components. Be careful with the default generated format from the EDA. It should be adjusted to the assembly house requirements.
	\item Component \textbf{location} and \textbf{orientation}. For KiCad, this is satisfied by the "CtrYd" (Courtyard) and the "Fab" layers respectively.
	\item Solder \textbf{paste} location. The paste data is used to manufacture an SMD stencil required for reflow soldering. It is a dedicated layer in the EDA tools (for KiCad "Paste" layer). The solder area is almost equal to the size of the pad.
	\item For automatic assembly, specific position files readable by Pick and Place machines are needed, the so called \textbf{X,Y files}. These provide information about the component position and orientation (.pos extension file for KiCad).
	\item Files with location data for the tests points (\textbf{test fixture}), if testing is integrated in the assembly process.
\end{itemize}

%\href{https://www.seeedstudio.com/blog/2019/06/12/how-to-generate-assembly-files-and-why-they-are-important/}{source}

\begin{figure}[h!]
	\centering
	\includegraphics[keepaspectratio, height = 0.25\textheight, width = \textwidth]{assets/solder_paste_stencil.jpg}
	\caption{Stencil by \href{https://www.itmconsulting.com/?product=stencilpro-3-0-stencil-aperture-calculator}{source}}
\end{figure}

\section{Identifying the design}

%Distinguish the lumped model and the distributed model? When we need to worry about transmission line theory? When circuit theory isn't adequate for designing? What about the digital spectrum, not sine waves? Rule of thumb related with wavelength and propagation speed? Logic levels and rise time? Bandwidth, skew rate. Should I keep in mind these values for filter design and when to apply filters? Frequency response of the components selected must be reviewed? 

Let's start talking about the challenging part of PCB development cycle, the \textbf{design}. All the problems of the PCB can be categorized to \textbf{Signal Integrity} (SI): "involving the distortion of signals", \textbf{Power Integrity} (PI): "involving the noise on the interconnects and any associated components delivering power to the active devices" and Electromagnetic Compatibility\footnote{Usually Electromagnetic Interference (EMI) is referred to the cause of the problem and EMC is referred to the solutions}(\textbf{EMC}): "the contribution to radiated emissions or susceptibility to electromagnetic interference from fields external to the product". These definitions aren't very helpful and sometimes problems of one category may overlap with the another but we are going to investigate them more in the DFP section.

%In general for a long time interconnections of PCB isn't considerable for electrical performance but this changed as the frequency is getting higher. The question is when signal integrity becomes a problem? This is has no clear answer but in general if you are in the MHz region you should start taking high speed guidelines into account.

For now, it should be noted that it is very important to distinguish in the design the existence of transmission lines. They are a subset of the root cause of many problems regarding signal integrity (reflection and the associated side effects) and you will understand why soon.

\textbf{Introduction to distributed model and transmission line theory. }
Electromagnetism phenomena can be divided to 1) high power and low frequency (electric machines and plants, transformers) and 3) high frequency and low power (mobile devices, modern systems). The focus in this report will be on the second part, so let's take a look on why high frequency really matters? Is actually the frequency the root cause of circuit issues?

In the PCB industry, designers for a long time in the low-frequency era neglected the \textbf{electromagnetism} and build their designs according to its \textbf{abstract} and less complicated version, the \textbf{circuit theory}. The so called traditional \textbf{lumped} model is based on principles such as 1) everything can be modeled via resistors, capacitors, inductors 2) everything happens instantly (there aren't time transients) and 3) interconnections that connect components are equipotential surfaces. This abstract model is a very efficient and practical method to make your application functional without worrying with the complex nature of electromagnetic fields. But as technology is progressing and things are getting smaller and signals are getting faster, the electromagnetic phenomena can not be neglected anymore. This is where \textbf{transmission line theory} along with the wave properties are coming to the surface, that connect the electromagnetism with the circuit theory and introduces the \textbf{distributed} model. According to this model, circuit theory can also be a great tool in high speed applications but with some modifications. Now the tracks that carrying signals can't be treated like equipotential surfaces but as an RLC circuit that extends as a function of the length of the track (image). This result to a much different electric behavior and should be consider. 

%But of course this isn't the only thing that needs to be addressed and progressively more concerns will be mentioned (check design for performance for high speed applications).

%Trying to explain what is reflection, what is radiation. What are the basic concepts of transmission line. Reflection coefficient, standing waves and so on. I am not sure if I am good at this. Maybe wikipedia will help for the generic stuff that I am interested. Interesting thing about antenna impedance is that the goal of the antenna is to have real part and no imaginary part that actually store energy. This happens in fractions of wavelength. The antenna is in resonance. That is why impedance transmission are real. But also 

%\textbf{A very quick introduction about transmission line theory and electromagnetic radiation}.
This different electric behavior is related with the fact that signals are starting to act like waves. But why and how?
Waves in nature have some fundamental characteristics. The \textbf{wave} nature of voltage signals is starting to take shape when frequency is very high and length of the interconnections is such that fractions of wavelength can fit to the conductive traces. Just like ocean waves and ropes reflect when they meet an obstacle, a voltage signal can potentially reflect back to the source. A very nice \href{https://www.youtube.com/watch?v=DovunOxlY1k}{video} by AT\&T can help a lot with the visualization to consolidate the analogy! These reflections is the cause of the important aspect of impedance matching. In the PCB case, they can be observed when there are discontinuities (impedance mismatch) regarding the geometry of the conductive path along the way that the signal propagates. Another feature that needs to be considered is the \textbf{radiation} aspect. When charges accelerate, electromagnetic fields are taking shape. The higher the frequency, the shorter the rise time, the faster the acceleration, the greater the radiation. The goal of the transmission lines is to guide the energy with maximum power transfer, zero reflections to an antenna and without radiating energy. So we could summarize that in high frequencies, voltage signals behave like waves and the charge acceleration radiate electromagnetic energy that it can be coupled to undesired pathways if we don't pay attention. We will mention in the section PCB guidelines, more about the problems and the solution regarding high speed signals.

But why transmission lines can be bad? What are the transmission line effects? They can cause timing issues due to the delay of the signal to reach from the driver to the source (tuning traces) and reflection issues such as false triggering, ringing, more EMI and crosstalk, overshoot, attenuation etc. As it seems transmission lines can affect severely the signal integrity so we need first to identify them and secondly to address the problems of reflection and delay with impedance matching, proper termination and tuning traces (section DFP).

%There is always radiation. The problem is when it is significant. Energy of radiation is proportional to frequency and inversely proportional to wavelegnth. The shorter the wavelength, greater radiation. The best way to fight the radation is the retunr path for field cancellation.

%In the best case scenario transmission lines should not radiate. To achieve something like this the reference signal, the return path should be close to the signal to cancel the EM fields.

%Don't forget to include why transmission theory isnt bad, antenna and the applications! Reflection cause ringing, radiation can couse coupling and son and these are negative side effects in signal integrity. Why is bad and why is good. As the frequency is getting higher then the EM waves are like dogs that are burking and they can't wait to unleash them to run. The goal is to guide the EM waves and not let them distribute to unintentional pathways. The goal is to guide energy!

Sometimes we are very good at learning individual things but connecting the dots is equally important. A recommended book to connect electromagnetism and circuit theory about electronics is by Ralh Morrison "Fields and electronics". It is very important to start thinking in fields and space rather traces that carrying signals. Viewing traces as the boundary of the space, is more helpful as a designing mindset in the AC world.

%This paragraph is more design oriented, not suitable for the introduction. The impedance matching is referred to such and such. (Be careful when changing the medium!) Check all about circuits for sure. But why reflections are actually bad? Because a portion of the desired energy will be eventually delivered to the load causing circuit failures (logic failures in the digital world). The load didn't receive at a particular time frame the expected voltage. The reflections that are going back and forth may not attenuate until the beginning of the next input signal causing many problems, is this bouncing effect?

\subsection{To be or not to be a transmission line}

A very important question that a designer should ask in the early phase of the decision making about the designing strategies is \textbf{when should I worry about transmission line theory?}\footnote{Transmission line mainly can be referred to any pair of signal and return conductors. In this case we worry about the reflection aspect of transmission lines.} To answer this question we need to know: 1) The propagation velocity of the signals of interest, so the constant of the material that the EM fields are propagated through. In our case we need the dielectric (typical FR-4) one of the PCB. 2) The frequency of the signal that can determine afterwards the wavelength. Wavelength is quite interesting because it is referred to an oscillation, a wave, a sinusoidal function. Circuits are often described by how they respond to sine waves of various frequencies (by Ralph). What happens to the non-sinusoidal inputs though (e.g. square waves)? How to analyze them? 

%Should we also distinguish analog and digital signals? Are analog signals waves, it is a signal and what is its frequency spectrum? We will stick to the wave analysis and the non sinusoidal inputs. For analog the wavelength is the criteria and for analog the rise time!



%A list with different rule of thumbs about when to worry about reflection and stuff:
%\begin{itemize}
%	\item Page 21 by Transmission line analog and digital, divide by ten
%	\item Altium live RF series, I think something about twenty
%	\item Managing signal integrity, use time for reference. Dont forget to mention comparable 
%	\item Ralph Morisson? Actually no rule of thumb, only how to approximate square wave with continuous wave, but I don't see this very often.
%	\item Eric Bogatin? page 595
%	\item Wikipedia claims divide by ten
%	\item Daniel Beerek by 7 in his live thing
%	\item Gioultsis by 100.
%	\item Mark montrose divide by 20.
%	\item circuit companion divide by ten
%\end{itemize}

%We are going to summarize the above with the following statement:
%This rule of thumb should not be taken too seriously. Designer may faced transmission line effects even in cases of $\lambda/40$ or can ignore them until the length if comparable with $\lambda/4$. It depends on the acceptable noise margins, the sensitivity of the circuit and the tolerances. So the best response for when designers should worry about these effects is \textbf{When these become important to the design}. In other words the real issue that needs to be considered is how much reflection can the circuit tolerate and still be functional? 

First about the sinusoidal ones, for the transmission line model (microstrip, depicted figure 3), using the equation $ c = \lambda*f$ and knowing the frequency we can calculate the wavelength in free space ($c = 3*10^8$, the speed of light). Including the dielectric constant of the medium the wavelength of interest is $\lambda_d = \lambda/\sqrt{\epsilon_r}$\footnote{This formula is valid in a stripline environment that the field is surrounded by the dielectric. In the microstrip one the effective dielectric should be used due to the air. Propagation velocity of stripline is a stricter rule than the microstrip though.}. For the typical dielectric choice FR4 in PCBs, the constant is $\epsilon_r = 4.5$ (velocity factor, $1/\sqrt{\epsilon_r} = 0.47$).
% source for the above is circuit companion page 46, also this \href{https://www.youtube.com/watch?v=bVdwu1IoX4k}{guy} does the same thing
%sometime about propagation velocity they include the permeability. But this is 1, according to \href{https://blogs.mentor.com/hyperblog/blog/tag/velocity-of-propagation/}{this}
It is worth mentioning that \textbf{it isn't the frequency} that really matters, it is about comparing the length of the tracks that carrying the signal (voltage) with its wavelength. When these two are "comparable", then the designers need to incorporate in their thinking the transmission line theory. To define what is comparable, a rule of thumb well know in the industry with some differences among the references is that the length of the line should not be under $\lambda/10$! For example for a 100MHz signal with FR4 as medium, the length of the tracks should kept under 0,141 m for a transmission-free design! \textbf{But this rule of thumb should not be taken too seriously}. Designer may face transmission line effects even in cases of $\lambda/40$ or can ignore them until the length is comparable with $\lambda/4$. It depends on the acceptable noise margins, the sensitivity of the circuit and the tolerances. So the best response for the question when the designer should worry about transmission line effect is: \textbf{When these effects become important/noticeable to the design}. In other words the real issue that needs to be considered is how much reflection and coupling can the circuit tolerate and still be functional? 
%But don't take this rule of thumb too seriously, you need to include the circuit tolerance too? page 595 for eric bogatin. An overview of the acceptable noise margins of the logic families and the circuit tolerances will determine how much reflection can handle and the circuit still remains functional. In other words, \textbf{how much reflection can my circuit tolerate and still be functional?} 

%\textbf{Distributed element circuit} is the term that you want to design microwave capacitors and inductors using only copper, stubs and so on without passive components.

%About the distributed model wiki and \href{https://www.allaboutcircuits.com/technical-articles/transmission-lines-from-lumped-element-to-distributed-element-regimes/}{all about circuits} claim that isn't only about the wavelength and the reflection, but it is about the tolerance of the devices. A reflected small percent of the initial may be a very important difference to the operating device. So it depends on the circuit sensitivity.

%For the non sinusoidal, we need to define rise time, skew rate, slope and so on and logic levels. Then the road is better.

\begin{figure}[h!]
	\centering
	\begin{minipage}[b]{0.4\textwidth}
		\includegraphics[width=\textwidth]{assets/microstrip.jpg}
		\caption{\href{https://www.researchgate.net/publication/337629930_Design_of_24_GHz_Microwave_Bandpass_Filter}{source}}
	\end{minipage}
	\hfill
	\begin{minipage}[b]{0.4\textwidth}
		\includegraphics[width=\textwidth]{assets/distributed_model.png}
		\caption{\href{https://en.wikipedia.org/wiki/Primary_line_constants}{source}}
	\end{minipage}
\end{figure}


\subsubsection{Digital signals}

Non sinusoidal signals can be analyzed as the sum of sinusoidal ones (fourier transfrom). This is a very powerful concept that can connect the sine wave analysis with any signal of interest. A very common non-sinusoidal signal with great interest in the digital world is of course the \textbf{square wave} (actually the trapezoidal) and this will be the base to understand the analysis for non-sinusoidal inputs.

\begin{figure}[h!]
	\centering
	\includegraphics[keepaspectratio, width = \textwidth]{assets/clock_signal.png}
	\caption{Typical digital clock by Transmission analog and digital book}
\end{figure}

%But treating a signal as very large sum (actually infinite) of sine waves is quite troublesome, maybe we can approximate it with a single sine wave, knee frequency and so on, \href{https://resources.altium.com/p/why-there-transmission-line-critical-length}{source}. About approximating the pulse response Ralph Morisson has something to say in the section about spiked and pulses. Also you could check the Daniel Beerek in the Altium Live, he said something similar.

The \textbf{response} of a pulse can be analyzed as the sum of the responses of the sine waves that constitute the pulse. With that in mind, a pulse response contain a large amount of information regarding a range of frequencies. So when dealing with such inputs, the analysis of the circuits is getting more complex because the designer needs to think not one sine input but a frequency spectrum. In the case that we want to determine when the length of the trace is critical for mitigation of electromagnetic phenomena, we will introduce the terms \textbf{rise time} (time domain) and \textbf{bandwidth} (frequency domain) (and \textbf{skew rate}).

What is rise time? Rise time usually referred as the time needed to go from 10\% of the amplitude of the signal to the 90\%. But why? I think I have some stuff written in the report check also the douglas brooks book.

In high speed design, it can be approximated by the 10\% of the clock period ((eric bogatin page 113)). In some cases like FPGA and ASIC, rise time can reach 1\%! However, wafers could be designed in such a way that can affect it dramatically despite operating in low frequency. It's a function of the logic family and the chip technology.
%(Rise time is what actually matters because this is where high frequency components exists and this will determine the threshold for the designing strategy of incorporating high speed guidelines to our design.) 
Another rule of thumb claims that it can be estimated as the\textbf{ 7\% of the period}. As we said, most of the times rise time will be 10\%, but is better to underestimate than overestimate. Rise time is essential as an input for design strategies and the why will become clear soon.


From the time domain we will jump to the frequency one. As we have said, a pulse can be described as an infinite sum of sinusoidal functions, but the \textbf{infinite} term is quite troublesome for the analysis. Do we need all the harmonics for an adequate representation of a pulse? We can actually neglect some frequency components due to the very low magnitude. In other words when they are not significant. Hence we can define \textbf{bandwidth}\footnote{The term bandwidth is used as the highest frequency component because in the frequency spectrum digital signals start at the DC} as the highest sine-wave frequency component that is significant in the spectrum. But how to define the "significant". There are some variations regarding this:

\textbf{First} approach by Clayton R. Paul. If we are going to plot the frequency spectrum of a trapezoidal signal we can observe that while the frequency increases, the magnitude decreases. Precisely in the figure 8 we see that after the breakpoint $1/\piτ$ the levels of harmonics are rolling off at a rate of 20dB/decade, then at the second breakpoint at a rate of 40 dB/decade. If we go past the second one by a factor of about 3 to the frequency that is the inverse of the rise and fall time, $f = 1/τ_r$ then the levels of the component we will be reduced further by 20dB. This breakpoint will be $f = 3 * 1/\piτ_r \approx 1/τ_r$.  Hence we can claim that above this frequency the components will be negligible and we define the bandwidth of the trapezoidal clock as:

\[\text{BW} = \frac{1}{τ_r}\]

\begin{figure}[h!]
	\centering
	\includegraphics[keepaspectratio, width = \textwidth]{assets/magnitude_clock.png}
	\caption{Bounds on the spectral coefficients of the trapezoidal pulse train for equal rise and fall times}
\end{figure}

\begin{figure}[h!]
	\centering
	\includegraphics[keepaspectratio, height = .4\textheight, width = \textwidth]{assets/square_n_trapezoidal.png}
	\caption{By eric bogatin}
\end{figure}

\textbf{Second} approach by Eric Bogatin. Let's try to re-create an ideal square wave (zero rise time) by adding harmonics. We can see in figure 10 that whenever we add more harmonics, the rise time becomes shorter and a trapezoidal signal is starting to take shape. Eventually the trapezoidal could become an ideal square if infinite harmonics were added. We are not interested in an ideal square, but in the trapezoidal ones that resembles the digital signals. So, for these signals, from which frequency and after, adding harmonics isn't significant anymore? Let's take the harmonics of an ideal trapezoidal signal and a square one. We are looking for the harmonic of the trapezoidal that its power is 50\% less than the power of the same harmonic of the square wave or when the amplitude is 70\% less. Then this harmonic is the highest frequency component and thus the bandwidth is defined.

\begin{figure}[h!]
	\centering
	\includegraphics[keepaspectratio, width = \textwidth, height = .3\textheight]{assets/harmonics.png}
\end{figure}

So we are actually using the harmonics of a square wave to build a trapezoid signal but the the highest frequency component is found by comparing the amplitude or the power of the harmonics of a trapezoidal and an equivalent square wave. According to this there is a linear relationship between bandwidth and rise time depicted in figure 11. Mathematically can be described as:

\[\text{BW} = \frac{0.35}{\text{Rise time (10\%-90\%)}}\]

\begin{figure}[h!]
	\centering
	\includegraphics[keepaspectratio, width = \textwidth, height = .3\textheight]{assets/band_rise.png}
	\caption{Linear relationship between bandwidth and rise time of trapezoidal signal}
\end{figure}


\textbf{Third} approach by Howard Johnson. Let's take the power spectral density of the signal depicted in figure 11. We can see that from the clock frequency until the so called knee frequency we have a 20dB/decade slop. Beyond knee frequency the amplitude rolls of faster. At this particular breakpoint the spectral amplitude is down by half (-6.8 dB) below the 20dB/decade slop. Thus we can claim that most of the energy in digital pulses is concentrated to the frequencies from the DC to the knee frequency. How to find the knee frequency based on the rise time, though? There is this formula:

\[F_\text{knee} = \frac{0.5}{\text{Rise time (10\%-90\%)}}\] 


\begin{figure}[h!]
	\centering
	\includegraphics[keepaspectratio, width = \textwidth]{assets/eye_spectra.png}
	\caption{Eye diagram of a digital signal and its power spectral density}
\end{figure}

With these three methods we are able to measure the highest significant frequency component of a digital signal. All the sine waves from DC to that frequency is important for our design, so the transmission line should be able to handle this range of spectrum. If a portion of this frequency range falls of to the category of transmission line as it was defined with the rule of thumb for the wavelength per sine wave, then high speed design guidelines should be definitely considered. 

\subsection{A transmission line problem}

Let's see an example to comprehend the transmission line problem in digital signals. We will be based to the first approach to estimate the bandwidth. So, let's assume that the voltage source depicted in figure 12 is a clock waveform of 5V, 50 MHz, 50\% duty cycle, rise time 0.5ns and the length of the interconnection is 2 in (0.0508m). The bandwidth of this waveform is

\[\text{BW} = \frac{1}{0.5 * 10^{-9}} = 2 \text{GHz}\]

Let's calculate what should be the frequency of a sine wave in order for the interconnection to act like an electrical short according to the rule of thumb of $\lambda/10$. So we set $\lambda/10 = 0.0508 \text{m}$. We assume FR4 as dielectric (($1/\sqrt{\epsilon_r} = 0.47$)), so

\[\lambda/10 = \frac{c}{f * \sqrt{\epsilon_r} * 10} = \frac{3 * 10^8}{f * 10} * 0.47 \rightarrow f = \frac{3 * 10^8}{0.0508 * 10} * 0.47 = 277 MHz \]

We have a signal that its frequency clock is 50MHz and the bandwidth 2 GHz! For the range 277MHz - 2GHz the interconnection isn't electrical short and the distributed model should be introduced. So definitely for this, transmission line effects should be taken into account. As you can see the clock frequency of 50 MHz doesn't tell much for the criteria to be or not be transmission line... 

\begin{figure}[h!]
	\centering
	\includegraphics[keepaspectratio, width = \textwidth]{assets/transmission_input.png}
	\caption{\href{https://incompliancemag.com/article/an-overview-of-transmission-lines-in-electronic-systems/}{source}}
\end{figure}

In a different way we could calculate at which length of the interconnection you should start worrying and similarly we have (f = highest frequent component, BW = 2 GHZ):
\[\lambda/10 = \frac{c}{f * \sqrt{\epsilon_r} * 10} = \frac{3 * 10^8}{2*10^9 * 10} * 0.47 = 0.007 \text{ m} = 7 \text{ mm} \]

Thus for length below 7 mm, we can estimate that the interconnection won't behave as a transmission line.

\subsection{Lumped vs Distributed}
\textbf{Another method} proposed by Howard Johnson for identifying \textbf{lumped vs distributed systems} is the following:

Everything takes times and nothing can travel faster than the speed of light. So the signal from the driver to reach the load will take some time, there will be a delay. This delay is proportional to the length of the trace, but during the delay there is a possibility depending the rise time to fit a portion of the signal to this segment of trace and wave nature is coming to the surface. The rising edge is the factor that determines if the signal can fit to the segment of the trace. We can defie the length of the rising edge (in m) as:

\[l = \frac{T_r}{D}\]

where $T_r$ is the rise time (10\%-90\%) in ps and D is the delay, how much time took for the signal to move per unit length (ps/m).\textbf{ For circuits smaller than $l/6$ are lumped}.

\begin{figure}[h!]
	\centering
	\includegraphics[height = .3\textheight, keepaspectratio, width = \textwidth]{assets/rising_edge_length.png}
	\caption{Length of the interconnection and the length of the rising edge}
\end{figure}

%It should be noted that high frequency harmonics attenuate significant more than the lower ones due to conductive and dielectric loss especially when traces are long. Less high frequency components, the rise time is increasing and the bandwidth is decreasing, resulting in a much different wave that the initial one. 

%\subsection{Summary}

%The sharper the edge, the shorter the rise time, the bigger the bandwidth, transmission line effects are emerging. Transmission line phenomena are related with reflection and radiation (coupling, crosstalk). How these can actually affect the design and what we can do about them? We will see in the next section.

%more radiation and emissions. The following equation is an approximation and it is based on\textbf{ 50 percent duty cycle}, rise time and frequency of a square pulse. The rise time is the fundamental contributor of the digital spectra and not the frequency of the digital signal itself. 

%But why having transmission line is bad? Actually it isn't bad at all, there are many applications based on the reflection and the radiation. But the point is to identify these characteristics and mitigate them when it is needed or make use of them. In our case, we don't design antennas so we need to suppress radiation and guide the field where it is supposed to go, to the desired load.

%Our goal is to not have high frequency components to avoid reflections and radiation or to lower their magnitude to become insignificant. For example if we don't take into account impedance matching to mitigate reflection, the ringing will cause to make negligible high frequency components to significant.

%Nice to know: The energy associated with a wave is directly proportional to its frequency. Hence, the higher the frequency, the shorter the wavelength and the higher the energy of the wave. It is all about the energy of radiation, amplitude and frequency.


%\section{PCB design guidelines or challenges, issues, problems and solutions}


\section{Design for Performance}

%As the frequency gets higher the impedance is dominated by reactance and not resistance. Smaller loops can significantly reduce the inductance, so the return current path following the rule of the least impedance, will travel below the signal trace. That is why slots and switching layers is the root cause of many problems. If trace of the length is one time important then making the loop area is ten times important (\href{https://learnemc.com/pcb-layout}{source})

%In the case of transmission lines should be also consider the impedance matching and tuning the traces for time critical signals. I am not sure if high speed is and distributed model is equivalent to transmission line. In other words if I keep my interconnects very short to meet the above criteria, do I need not to worry about nothing, not even planes and so on. Common mode noise is radically different than these but what else?

%In this section (Design for Performance\footnote{I am not really sure if this is an accurate term regarding the PCB literature. The purpose is somehow with this term to group most of the PCB electrical issues.}) we will try to give an overview of techniques regarding the electrical performance of a PCB design for the purpose to meet the requirements and the specifications, especially when we are dealing with high speed constraints. 

\textbf{Design for Performance} is referred to electrical performance and it becomes crucial in high speed applications when a lot of guidelines should be implemented to meet the requirements. 
High speed design is something that you can't explicitly define. We mentioned some rules of thumb for the transmission line and the distributed model, but in general in the MHz region and above you should treating lines with a careful mindset\footnote{ It is should be noted that the voltage of the signal is quite important for the radiation. A fast rising time of a 5V signal would have more energy than a 3.3V. This is why the \textbf{slew rate} (how fast the voltage change) can be actually a better indicator of signal criticality.}. It should be noted that every pair of signal and return path can be defined as a transmission line. We may keep the interconnections small and not having reflections and this is good but just a subset of the signal integrity.

\textbf{Question}: Even if I have very short rise times that means trouble, but the pin changes its state not very often (it isn't periodic like the clock that change its time very frequently), can I assume that it isn't a critical signal eventually? \href{https://www.ti.com/lit/an/szza009/szza009.pdf}{source, page 2, PCB design guidelines for reduced EMI}

% Most of the problems in PCB design fall of to these categories:
First we categorized the problems in SI, PI and EMC. A more practical way to categorize the issues in the design is the following:

%By Eric Bogatin:

%\begin{itemize}
%	\item Reflections
%Ringing, enhance the crosstalk, magnitude of higher components become significant
%	\item Crosstalk
%	\item Ground and power bounce as a special case of crosstalk
%	\item Losses. This is referred mostly to the GHz region and we are not so much interested to so high frequencies for this report.
%	\item Rail collapse in the power and the distribution network
%	\item EMI
%\end{itemize}

By Douglas Brooks

\begin{itemize}
    \item \textbf{E}lectro\textbf{M}agnetic \textbf{I}nterference (radiations beyond the board or susceptibility to radiations from outside the board)
    \item \textbf{Reflections} on a single net
	\item \textbf{Crosstalk} between two or more nets, in many ways a special case of EMI
	\item \textbf{Power system stability} 
\end{itemize}

\subsection{Forget the Ground think Return}

%Eventually the current should return to ground and will try to find a bypass capacitor to do that (\href{https://resources.altium.com/p/should-you-use-your-power-plane-as-a-return-path}{source})

%In high frequency what determines the impedance is not the resistance but the reactance.

In high frequency signals where voltages and currents changing/oscillating, there aren't any shorts or opens like we are used to. Even two conductors with voltage difference separated by a dielectric can act as a very low impedance path and definitely not something that current can't flow (displacement current). Another thing that we need to understand is that \textbf{current flows in loops} and always trying to find the path with the \textbf{lowest impedance} (yes impedance and not resistance. In higher frequency reactance dominates resistance, \href{https://learnemc.com/pcb-layout}{source}). This path will be the adjacent copper area because the loop is smaller, the capacitance is higher thus the impedance is lower. So it is considered best practice to place the ground plane (about planes check this) very close to the signal.

\begin{figure}[h!]
	\centering
	\includegraphics[keepaspectratio, width=\textwidth, height=.2\textheight]{assets/signal_return.png}
	\caption{By eric bogatin}
\end{figure}

But if we think that everything is fine by connecting the return path to the ground plane, this is where a lot of problems arise. As we have said the current will find the path of the lowest impedance. If for some reason there is copper area between the signal and the ground then the return path will not be in the ground but in the copper between them. In other words the current will follow a radically different route from what we estimated with side-effects such as overlapping currents, crosstalk, distortion! In the end the current should return to ground. So a very important rule is to have an \textbf{an unbroken dielectric} between the signal and the return. 

\begin{figure}[h!]
	\centering
	\begin{minipage}[b]{0.4\textwidth}
		\includegraphics[keepaspectratio, width=\textwidth]{assets/broken_dielectric.png}
		\caption{By eric bogatin}
	\end{minipage}
	\hfill
	\begin{minipage}[b]{0.4\textwidth}
		\includegraphics[width=\textwidth]{assets/side_current.png}
		\caption{by eric bogatin}
	\end{minipage}
\end{figure}

\subsubsection{Changing trace layers}

And if I have to change layers how to follow the rule of the unbroken dielectric? What will be the return path? This is quite interesting actually and in order to visualize it let's inspect the figure 14.	In this figure we assume that layer 2 and 3 are reference planes with the same potential, thus we can connect them using a via. But before talking about the via, we can observe that in the layer 1 and layer 3 the return current follow the principle of the adjacent layer as we expected. The big challenge is what is happening between the 2 and 3, how the current can return? Thus we put the via to provide a path for the return current to transition from the 3 layer to the 2.

Let's now see the figure 15, without using a via and having as planes the ground and the power for the 2 and 3 correspondingly. How the current will return? It will spread out around the clearance created from the hole and will try to make use of the capacitive coupling between the two planes. Why it spreads? For lower inductance and for higher capacitance. This discontinuity though it will increase significant the impedance causing a voltage drop in the ground plane the so called ground bounce. For this case we could place a bypass capacitor near the via that will act as a low impedance path for the current (be careful with the loop inductance, the bypass capacitor eventually may not act as one). We could also minimize the clearance of the hole, making the via smaller to reduce the voltage drop. The distance between the two plates in order to increase the capacitance and lower the impedance should be kept as small as possible.

\begin{figure}[h!]
	\centering
	\begin{minipage}[b]{0.4\textwidth}
		\includegraphics[keepaspectratio, width=\textwidth]{assets/change_layer.png}
		\caption{\href{http://www.sigcon.com/Pubs/news/6_04.htm}{source}}
	\end{minipage}
	\hfill
	\begin{minipage}[b]{0.4\textwidth}
		\includegraphics[width=\textwidth, keepaspectratio]{assets/ground_trans.png}
		\caption{\href{https://www.tempoautomation.com/blog/design-to-avoid-emi-problems-keep-clocks-away-from-unintended-antennas/}{source}}
	\end{minipage}
\end{figure}

\begin{figure}[h!]
	\centering
	\includegraphics[width=\textwidth, height=.25\textheight]{assets/change_layer2.png}
	\caption{by eric bogatin}
\end{figure}


%The current can't penetrate between the two planes (capacitance between plates isn't enough) and will try to find a way to come back to the ground, probably by a bypass capacitor locates somewhere in the board. A better approach is to provide a return current path close to the one the via that drive the signal to another layer either with an additional via (ground transition vias) if the planes have the same voltage or with a bypass capacitor.

\subsubsection{Slots}

Another thing encountered in PCB designs is a disrupted return path with slots in the ground plane. The return path is disrupted, the current will flow around the loop, thus the inductance increase resulting to more emissions and voltage drop (ground bounce). The field of the signal path during the slot isn't canceled by the return path. To calculate the inductance of these slots for accuracy you could use 3D field solvers. In general the bigger the area of the loop, the higher the inductance and the more the problems. It would be best to avoid slots or don't route signal over them or if you can't do anything you could place a bypass capacitor as a path of low impedance (\href{https://electronics.stackexchange.com/questions/81761/whats-radiating-on-my-pcb}{source})

\begin{figure}[h!]
	\centering
	\begin{minipage}[b]{0.3\textwidth}
		\includegraphics[keepaspectratio, width=\textwidth]{assets/slot.png}
		\caption{\href{https://www.tempoautomation.com/blog/design-to-avoid-emi-problems-keep-clocks-away-from-unintended-antennas/}{source}}
	\end{minipage}
	\hfill
	\begin{minipage}[b]{0.3\textwidth}
		\includegraphics[width=\textwidth]{assets/slot_via.png}
		\caption{Vias clearance cause slots in the ground plane. High speed layout guidelines TI}
	\end{minipage}
\end{figure}

Also these type of slots can be created by holes too like vias. If vias are placed very close to each other the clearance will create a gap without any copper between them. This will result to higher impedance that is of course undesired. So be careful with the clearance of the holes. Make them smaller or further the distance.

Now that we analyzed the slots we will introduce another very important rule like the unbroken dielectric, \textbf{the unbroken return path}.


\subsubsection{Summary}

What should I keep? Two important rules for high speed design: 1) Unbroken dielectric and 2) unbroken return path.
%As we have seen , \textbf{identifying the return path} is game changer for mitigating a lot of issues. 
To put it in another way, you should remember: \textbf{IT IS ALL ABOUT THE FIELDS}. Start thinking in pairs of conductor and dielectric, not per signal trace.

Identifying the return current can solve a lot problems. Where is the source, the load and the return.

%The aforementioned guidelines are a portion of the solutions to mitigate signal integrity regarding the EMI and the rail collapsing noise. More will be mentioned all the way.

The aforementioned guidelines fall of to the category of solutions to mitigate EMI and rail collapsing noise (subset of power integrity).

\subsection{Reflection}

As we have said in the section about the design identification, the signals can have wave nature. Precisely, we analyzed when these kinds of reflections should be significant by comparing the wavelength of the signal with the length of the interconnection. In the case we have transmission lines and reflections, what should we do to avoid problems such as ringing, false triggering, crosstalk and distortion? The answer is \textbf{impedance matching}.

Reflections happens each time the signal face an impedance discontinuity along its way of propagation. These discontinuities can be via, junctions, traces with different geometric shape, connectors, pins of ICs etc. Our goal it to shape a path for the signal with a constant impedance along the way. But we need first to define what is \textbf{characteristic impedance} for a transmission line. It is the constant, instantaneous impedance that the signal looks. It can be defined as the input impedance of an infinite transmission line. It should be noted that there are many formulas that can calculate the characteristic impedance of any type of transmission line. In our case we are mostly interested in microstrips.


\begin{figure}[h!]
	\centering
	\includegraphics[width = \textwidth, keepaspectratio, height=.3\textheight]{assets/impedance_matching.png}
	\caption{Eric Bogatin}
\end{figure}

Our goal it to match the impedance of the transmission line with the source impedance of the driver and the load (but actually only for the impedance of the interconnection we have control). Let's suppose that we have a driver, a load and an interconnection (a uniform transmission line without stubs) and we want to design the transmission line so there aren't reflections. First we need to know the source impedance and the load impedance. For a typical CMOS device that drives the signal we suppose that we have impedance of 5-20 Ohms. Similarly for the load we suppose that we have a capacitor. In most CMOS reveivers the capacitance value is very low so it can be approximated as an open circuit. The source impedance can be extracted by IBIS models or by datasheets too. So how we do the impedance match? 

Usually the microstrip traces are routed in such a way adjusting the width to have characteristic impedance \textbf{50 Ohms}. In RF application, with coaxial cables and such, the traces on the PCB are designed with a 50 Ohm characteristic impedance. This is a well know standardized value for cables and RF chips. It is a convention in order to help different vendors to design their products having this value in their minds for impedance matching between different products. But in the case of PCB tracks, the density is very high and designing 50 ohms trace isn't often very convenient. If the design isn't dependent of a coaxial cable or something that forces the trace to be 50 Ohms, can I still use a different characteristic impedance?
 
The goal it to match the impedance. So if we had a 100 ohm characteristic impedance and the source impedance is 10 ohms, by connecting a 90 ohms resistor in series with the driver then we have impedance matching. However with this configuration, we created a voltage divider and the voltage waveform in the transmission line will be actually half of the intended. But due to the 100\% percent reflection\footnote{When a transmission line is opened from the load end then all the forward signal is reflected back to the source} at the end of the transmission line, the total waveform by adding the reflected will be V volts and when the reflected wave reach the source it will not be reflected back because of the 90 + 10 = 100 impedance.

It should be noted that terminations and impedance matching is a quite huge topic (we didn't even scratch the surface!). It is recommended to read also the documents provided by the vendors that will suggest the best way for the termination of the critical signals. For example, regarding the high-speed USB, the impedance for the differential pair should be 90 ohms.

%\textbf{For purposes of this explanation, CMOS receivers look like very small capacitors that can be considered to be open circuits}, \href{https://www.altium.com/solution/transmission-lines-and-terminations-in-high-speed-design}{source}

% PCB tracks not 50 ohms, \href{https://electronics.stackexchange.com/questions/325143/are-i-o-buffers-of-ics-design-to-have-50ohm-impedance}{source}

%What do we mean that the signal will be absorbed?

%\begin{figure}
%	\centering
%	\includegraphics[width=\textwidth, keepaspectratio, height=.3\textheight]{assets/terminating.png}
%\end{figure}

%If the impedance of the source is a resistor? It is typical for CMOS? The transmission line acts like a voltage divider, so the lower the resistance the more voltage will be applied to the ends of the transmission line. Yes usually CMOS drivers have source impedance and the signal is modeled with a voltage source. Of course we have some kind of voltage drop due to the resistor, not everything go to the transmission line.

If you want to connect multiple receivers then daisy chain routing techniques is recommended in order to keep the characteristic impedance controlled without making branches and stubs.

\textbf{Question}: If I have a via along the interconnection path? What should I do?

\subsection{Crosstalk}

%What are the requirements? How much noise does the board can handle? Interconnection density ? Everything comes with a cost! What are the acceptable noise margins?

%Fringe fields, field that spread, is this far field vs near field

%Coupling happens only when things change.

%You could mention the 20H rule too.

%If we are close to the fringe field region, when voltage, current change then this will cause current to flow through the changing electrics as displacement current and as induced currents from the magnetic fields. 

%Our goals is to minimize the overlap of the fields between two signals: 1) minimize the space between the return path to reduce the spread and further the distance between the signals.

%EM field intensity is inverse proportional with the square root of distance, so the further apart the better.

%It is very important to translate the geometry of the traces into capacitive and inductive noise. Simulation!

%Of course crosstalk get worse if we have reflections without taking into account anything about impedance matching. Actually in general a lot of things get way worse.

When there is current there is magnetic field, when there are charges there is electric field. When current change, magnetic field change and this can induce current to nearby conductors if the magnetic rings spread by the source are shared with them. Because of this mutual inductance and the \textbf{inductive coupling}, noise is produced. This is called switching noise (because happens during the switching part of the current, during the rise time). Opposite current can be induced to the same conductor that created the current in the first place too. The amount of the voltage noised is determined by the total inductance (the total amount of the rings that surrounds the conductor) and of course the rate of change. This is the so called inertia, it takes time to build up current. On the other hand when charges are oscillating, they create changing electric fields that can kick charges of the adjacent conductors and current shows up (displacement). This is called \textbf{capacitive coupling} and it is based on the mutual capacitance between two conductors. All of the aforementioned regarding the induced voltages and current when EM fields change, can be described by the notorious 4 Maxwell equations.

The change of these EM fields is actually the root cause of everything. Thus coupling happens during the edges, the rise time. When the coupling is undesired we call it crosstalk. Our goal is to have high capacitive coupling and low total inductance for the signal and the return and low capacitive coupling and low total inductance to unrelated signals.

\begin{figure}[h!]
	\centering
	\includegraphics[keepaspectratio, height=.3\textheight, width = \textwidth]{assets/fringe_fields.png}
	\caption{Microstrips, fields overlap \href{https://www.signalintegrityjournal.com/blogs/4-eric-bogatin-signal-integrity-journal-technical-editor/post/402-pop-quiz-use-tight-or-loosely-coupled-differential-pairs-to-reduce-cross-talk}{source}}
\end{figure}

%As we mentioned in the previous subsections, the current is following the path of the least impedance and this path is located to the adjacent copper area of the signal. But actually let's take a closer look on what is really happening for the purpose to understand how to make use of this induction and to prevent it when it is undesired.

%Forward, backward corsstalk and near and far field.
In this kind of topic usually authors write about forward, backward and far and near crosstalk. We won't go into too much detail. The concept is that in an environment that we have two microstrips and a signal is propagating to one of them, then there are two directions that induced current can flow, the forward (the same with the signal that caused the coupling) and the backward (the opposite of the signal) and two types of noise, the near (lower in magnitude but last longer) and the far (higher in magnitude but lost shorter). The near is at the start of the trace close to where the signal started to propagate  and the far is at the other end where the signal reaches the destination.

%It should be noted that capacitive coupling is what makes the current to flow to adjacent layer. The inductive coupling is inductive noise, thus we always try to minimize the inductance. I am not sure if this is valid. But in the inductance we mention that in general is noise.

% when the return path is a wide plane, then the capcitive and inductive coupling have the same magnitude. When traces are used for the return, the inductive coupling is far more severre, because the inductance rises radically

In the bottom line there are 4 main things that determine the coupling:

\begin{itemize}
	\item Width.
	
%The wider the return path, the better, this is why planes are used for ground. Also the bigger the width of the signal, less the inductance and higher the capacitive coupling in the return path but also to the adjacent unrelated signals which is undesired. Space between them is a more important factor.

About the inductive part, the less the density of the current the less the inductance. So increasing the cross section of a conductor also leads to a lower self inductance. This is why ground plane has low inductance too. By increasing the width of a signal, the capacitive coupling between the signal and the plane is increased too.
	\item Space.
	
Bring the return plane as close as possible to the signal to minimize the area of the loop (field cancellation, lower total inductance, less crosstalk and emissions) and further the distance of unrelated signals as much as possible, considering the density of the interconnection and mechanical constraints. We know that field intensity is proportional to the inverse square of the distance.

But why by minimizing the space between signal and return is better? As we can see there is field cancellation and the total inductance is low. 

	\begin{figure}[h!]
	\centering
	\includegraphics[keepaspectratio, width = \textwidth]{assets/field_cancellation.png}	
	\caption{Field cancellation}, \href{http://www.signalintegrity.com/Pubs/edn/FieldCancellation.htm}{source}
	\end{figure}

	\item \textbf{3H rule}.
	
There is also the 3H (three times the H) rule that indicates the required distance between an aggressor and a victim trace. The H is the height of the dielectric.
	
	\item Coupling length.
	
Reduce the length that unrelated signals run in parallel and in general for all the traces make them as short as possible (low self inductance).

	\item Keep rise time high.
	
As we know it is all about the rate of change! The higher the rate, more problems occur overall. But it isn't something that in most of the cases you can change.
\end{itemize}

The interesting part of crosstalk is how to simulate it. Usually 2D field or 3D tools are used to calculate the capacitive and inductive coupling and then these measurements are integrated to the modeling process. 

But how much crosstalk is too much? In order to find this you need to define the noise margins and the tolerance of your design. Then with the help of superposition and with modeling you could approximate how much noise it will be coupled.

%\subsubsection{Summary}

%So what should I do to with a few words to minimize crosstalk? For the inductive coupling our goal is to minimize the total inductance (self and mutual)

%\begin{itemize}
%	\item For signal and return path, keep them as close as possible, as short as possible to minimize the loop and wide to reduce the total inductance and increase the capacitive coupling to the desired return path. The wide though conflicts with the stray capacitance in a coplanar environment.
%	\item For unrelated signals, keep them as distant as possible and again as short as possible and less density for less inductance. If we increase the width of adjacent conductors with unrelated signals then the undesired capacitive coupling between them will increase but the desired capacitive coupling between the signal and the return will increase. With a 2D field solver you can find the best case, but most of the time there are space constraints so it is better actually to reduce the width to increase the distance. The distance is the primary factor that makes difference. Also is capacitive coupling desired for the signal and the return? I mean yes because you define where the current should be. It is better to not capacitive coupling to anything else rather than the return, you have field cancellation too. Also there is conflict with the self inductance and the stray capacitance. Which is dominated?
%	\item Something very cool that should be noted at least for me is that the current created to the return path is due to capacitive coupling. The inductive coupling, is actually the induced noise and it is undesired. It is the intertia and it can potentially with the rate of change of current emissions, crosstalk.
%	\item Decrease the coupling length
%\end{itemize}

\subsection{PDN}

\textbf{Power Distribution Network} (PDN) is the root of the power integrity. It is a fundamental part of any product and it plays a very important role in the overall performance of the board. In this network are included every part that is related with power distribution like interconnections, planes, bypass capacitors, voltage regulator modules (\textbf{VRM}). This network is responsible to feed with the necessary amount of power each component and to cycle the return currents too. It is centralized, high frequency currents are running through through the power lines and noise can accumulate to these interconnections causing emissions and functional problems. So it is quite critical to have a clean power supply providing low inductance as much as possible. Planes and decoupling capacitors are essential tools for controlling the PDN.

\subsubsection{Planes}

\textbf{Ground}

Having a low impedance path (low inductance) where currents are flowing is very crucial when we want to minimize voltage drops and having a clean voltage difference. This is why the ground that acts as a reference plane should be a solid \textbf{plane}. Planes are dedicated copper layers that take all the space. The more the copper, the less the current density, the less the inductance. If indeed the return is on the ground plane, then no ground bounce problems! Additionally, the planes have the ability to contain the fields. The fields can't penetrate the copper.

Avoid ground loops, because the loop is susceptible to crosstalk, EMI due to the mutual inductance. Return to the ground with the shortest way by placing vias to the ground plane rather than routing ground traces.

\textbf{Power}

Power plane is recommended in order to provide a low inductance to build up the necessary charge as fast as needed when the switching transistors are calling for it (check the decoupling section to comprehend it). The plane assist the job of the decoupling capacitors.
% eric bogatin page 317

By using a power plane, there is also a capacitance formed due to these adjacent layers (planar capacitance). Of course it isn't enough to provide the required decoupling the design needs. So don't forget the capacitors. The shorter the distance between the planes, the better though. Also, if for some reason there are unintentional return currents in the power plane, the higher the capacitance between the power and the ground, the lower the risk for integrity (subsection changing layers).


%Current flow both in the two planes, but the ground is more important due to the reference act, \href{https://electronics.stackexchange.com/questions/342518/conventional-current-flow-and-ground-plane}{source}

\textbf{Question}: Should I use a ferrite bead to filter the power supply?

\subsubsection{Copper pours}

Let's suppose that we have a four layer PCB and we have already set two dedicated layers, the inner, for the GND and the power. Would it be beneficial to pour copper to the top and the bottom where traces and components exist and use via stiching\footnote{Via stiching is a technique of spreading across the board vias to tie electrically big copper areas} to connect the pour with the planes? Can nearby conductors cause undesired coupling in the pours? Am I gonna create an antenna? What about EMI?

\subsubsection{Clock signals}

Now that we mentioned about the importance of planes, let's check a special case of a critical high speed signal that is the heart of digital applications, the clock signal.

It is recommended not to use the same ground plane with the rest of the circuit but a local/isolated one. Clock signals are consisted of high frequency components. Running such a high frequency current in a huge copper area like a plane that can be comparable with half the wavelength, could result to a \textbf{center-fed patch antenna!}, \href{https://electronics.stackexchange.com/questions/15135/decoupling-caps-pcb-layout/15143#15143}{source1}, \href{https://electronics.stackexchange.com/questions/39136/competing-pcb-crystal-layout-recommendations}{source2}. Also don't route signals near the clock traces and avoid vias.

About resources, usually application notes can be easily indexed by "Best practices for the PCB layout of Oscillators".

%You can add a ground plane, but you should consider the stray capacitance, \href{https://www.nxp.com/docs/en/application-note/AN1706.pdf}{source} when choosing the load caps.

%Be aware of the capacitive coupling (\href{https://www.youtube.com/watch?v=t5phi3nT8OU&t=4940s}{source}). This is why we remove the ground, but for the cap loads or for signal integrity? 

%What is the return current of the crystall oscillator. Actually I cant understand the loop. Supply, crystall, mcu?


\begin{figure}[h!]
	\centering
	\begin{minipage}[b]{0.3\textwidth}
		\includegraphics[width=\textwidth]{assets/isolated_gnd.png}
		\caption{Isolated ground plane should be connected to one point with the main ground, \href{https://www.youtube.com/watch?v=t5phi3nT8OU&t=4940s}{source}}
	\end{minipage}
	\hfill
	\begin{minipage}[b]{0.3\textwidth}
		\includegraphics[width=\textwidth]{assets/layout_clock.png}
		\caption{\href{http://ww1.microchip.com/downloads/en/DeviceDoc/Atmel-8128-Best-Practices-for-the-PCB-Layout-of-Oscillators_ApplicationNote_AVR186.pdf}{source}}
	\end{minipage}
\end{figure}

\textbf{Question}: This discrete, isolated ground for the high speed signal of the clock traces can be used as a general technique for high speed signals?

\subsubsection{Decoupling capacitors}

% I will stick to the application note of TI, high speed layout guidelines. Decoupling is the same with bypassing
Another huge factor that contributing to the power integrity of your design is the decoupling capacitors. It is worth mentioning that there are also the bypass capacitors and these terms are used interchangeably. So the decoupling/bypass capacitors are used to clean the power line from noise (low impedance path) and to provide a short burst of energy when the fast switching request power to drive the current. That's why for the decoupling, in order to cover a wider range of frequencies that needs to be shunt, we pick more than two capacitors. One for low frequency and one for high. Most of the times the vendors will provide the necessary information about the values. Or is this for the energy bursts? No I think

%\href{https://electronics.stackexchange.com/questions/2272/what-is-a-decoupling-capacitor-and-how-do-i-know-if-i-need-one/2274}{resource}

As we can see in the figure 22, if we didn't place a bypass capacitor, the urgent need of transistors for the charge to build up isn't satisfied by the power supply due to the inductance, the inertia of the power line. So we need to use a temporary power supply until the time that the system can reach the desired amount of current. 

\begin{figure}[h!]
	\centering
	\begin{minipage}[b]{0.4\textwidth}
		\includegraphics[keepaspectratio, width = \textwidth, height=.3\textheight]{assets/bypass_charge.png}
		\caption{Charge requirements of switching transistors, by Doug Brooks}	
	\end{minipage}
	\hfill
	\begin{minipage}[b]{0.4\textwidth}
		\includegraphics[keepaspectratio, width = \textwidth]{assets/decoupling.png}
		\caption{Smoothing the signal, \href{http://www.learningaboutelectronics.com/Articles/Decoupling-capacitor.php}{source}}
	\end{minipage}
\end{figure}

Another important thing for the decoupling capacitors is to place them \textbf{as close as possible} to the power supply pins of the component that you are trying to decouple. To minimize the inductance and to provide the energy as clean and fast as possible 
%and to not let switching noise noise radiate, because they act as a low impedance path for noise as we mentioned.
. If you have two capacitors with different values then place the smaller one closer. An interesting article for optimizing the decoupling capacitors location can be found here, \href{https://www.allaboutcircuits.com/technical-articles/pcb-layout-tips-and-tricks-how-to-optimize-your-decoupling-connection/}{all about circuits}



\section{Parasitic elements}

Not everything is what it seems! So far we said about the capacitance and the inductance of interconnections, but this also includes passive components, leads, connectors and so on. So it is quite important to be sure that you made the right choice of values for the passive components for the range of frequencies that you are interested along with proper PCB layout. Related to this frequency response is the so called self resonance, where for example capacitor acts like a complete pure resistor.

\begin{figure}[h!]
	\centering
	\includegraphics[keepaspectratio, height=.4\textheight, width = \textwidth]{assets/real_cap.png}
	\caption{\href{http://www.sigcon.com/Pubs/straight/resonance.htm}{source}}
\end{figure}

Most of the times we are referring to the magnitude of the impedance like high and low, but we are not mention anything about the phase. Should I care about phase? The digital circuit will work, but the analog components usually care about a lot about the phase so keep that in mind. Things change when frequency varies and remember in digital circuits we have a bandwidth and not a single frequency. (by \href{http://www.sigcon.com/Pubs/straight/resonance.htm}{source})

\subsection{Component Placement}

Before implementing any of those guidelines, the first thing that designers do is to place the component. Placement is referred as if not the most then one of the most important parts of the design. Some guidelines could be:

\begin{itemize}
	\item Give room to breathe! Consider the density of the traces that will run across the board.
	\item Partition the design (RF, digital, analog). Group similar parts together.
	\item Connectors should be placed to the edge of the board away from circuitry.
	
Input/Output pins is a very common way for noise to be coupled on and off the board. So for this reason should be placed away from the rest of the circuitry, in the edge of the board, source: A brief annotated list PCB EMC Design Guidelines

	\item Components that are \textbf{thermal} critical should not concentrated in a the same area in order to avoid hot spots. So distribute them and don't place them near the edge, because the heat removal won't be the best. \href{https://www.allaboutcircuits.com/technical-articles/pcb-thermal-management-techniques/}{source}
	
	\item Decoupling caps as close as possible to power and ground pins.
	\item Check the \textbf{ratsnest}\footnote{Ratsnest is a bunch of air wire connections that show the distance of the connections in the PCB layout} and find the most efficient way for the traces to be short and with less via.
	\item Design for assembly. It is generally preferred to have the IC the same orientation but it isn't mandatory.
	\item Sometimes rotating the IC with an angle of 45 angle can help in the next steps of routing.
	\item Don't forget the mechanical dimensions. Components should not overlap.
\end{itemize}

Finally an objective/artistic comment is that if it looks good it will work!

\subsection{Routing}

How to connect something? Let's have an overview of the different ways of routing. \href{https://resources.pcb.cadence.com/blog/pcb-routing-topologies-demystified}{source} Namely we have:

\begin{itemize}
	\item Daisy chain
	\item Point to point
	\item Star connection
	\item Bus
\end{itemize}

\subsection{Mixed design}

%Guidelines for designs that include analog, digital and RF signals. Partitioning, but what this actually means?

How to approach design including analog, digital or RF components? Should I split the ground? Be aware the return paths, the slots and not to overlap currents from different functional groups.


%As the frequency gets higher the impedance is dominated by reactance and not resistance. But when this happen? We will propably find out. I mean everything has inductance and capacitance and they are function of the length and the frequency. So what is the equivalent rule of thumb? Like the transmission problem? Maybe there isn't because a lot of factors contributing to these. But let's see how eric bogatin measure the capacitance and inductance. Inductance and capacitance is for sure a function of the frequency. The reactance of the traces is inherent but it shows only when high frequency.

\subsection{Mounting holes}

%Do your remember the era and how much time you are devoted to demystify for the clearance, vias on the holes and so on? For the EM report that was as I can recall.

%But let me tell you my pain. The most important thing that needs to be addressed is design for performance and simulation aspect of PCB that we actually skipped in the PCB thing but I trust you that you can do stuff. You have 5 more days to do only Acubesat work. 5 August it is a necessity to working for uni and divide your time into Acubesat and uni until 15 August. Then we have only uni things. OKay, I think it is doable!

Why there is clearance on the mounting holes? Should I ground them? Should I put vias on them?

\begin{figure}[h!]
	\centering
	\includegraphics[keepaspectratio, height=.3\textheight, width = \textwidth]{assets/mounting_holes.jpg}
	\caption{Different types of mounting holes \href{https://atadiat.com/en/e-four-pcb-marks/}{source}}
\end{figure}

%\subsection{Decoupling capacitors}

%How to place the decoupling capacitors, via vs trace, inductance of plane, inductance of via and small tracks. Check this article . This may be an overkill for our design. Vias should be used when having planes, but not for intersecting the layers due to EMI, place ground transition vias though. What is the arguement for using not many vias? Ground loop, no. Cost in the manufacturing? Impedance matching? Breaking the planes if they are through hole vias?

%Decoupling vs bypass. Smoothen vs shunt the noise. 2 caps for decoupling indicate by datasheets. The voltage drop thing is for the current carrying the plates that can de-charge the capacitor and have voltage drop, thus higher capacitance the better. How to increase the capacitance? Spreading caps all over?

%Interesting articles for decoupling: \href{http://www.sigcon.com/Pubs/straight/resonance.htm}{1}, \href{https://resources.altium.com/p/should-you-use-your-power-plane-as-a-return-path}{2}, \href{https://electronics.stackexchange.com/questions/274346/bypassing-vs-decoupling-capacitors}{3}, \href{https://components101.com/articles/decoupling-capacitor-vs-bypass-capacitors-working-and-applications}{4}

\subsection{Unused pins}

Most of the times the datasheets will provide information what the designer should do about the unused pins, the floating pins of an MCU for example. The common solution is to tied them to ground, providing a low impedance path. But is this always the case? Is a myth to tie every floating pin to ground (\href{https://learnemc.com/not-so-good-emc-design-guidelines}{food for thought})?

%\subsection{2 Layer design}

%2 layer design is a special and very common in PCB. Some specific tips for this type of design are the following:

%\begin{itemize}
%	\item Horizontal traces above, vertical below.
%	\item Use the leftover area, make it ground
%	\item Be careful with slots. This is a general rule of thumb though
%\end{itemize}


\section{Design for Manufacturing}

In order for the PCB to transition from the Gerber state to reality, it should be manufacture-able! Therefore DFM (Design for Manufacturing) guidelines should be taken into account early at the design stage\footnote{The design stage is the most cost efficient stage to detect and fix potential problems} to avoid product failures and save time and money. 

Even if something can be fabricated, small changes in the layout can reduce the total cost. This isn't always the case for small quantities but at mass production scale, even the smallest nuances will increase it significantly. 
%\href{https://resources.pcb.cadence.com/blog/2019-product-development-for-electronics-and-hardware-planning-for-scaling}{source}. 

Usually EDA tools feature \textbf{DRC} (Design Rule Checking) tools, that automatically inspect the board to determine whether it meets the constraints imposed by the manufacturer.
%(for compliance with standards used by the manufacturer)
For instance, Eurocircuits offers some \href{https://www.eurocircuits.com/drc-settings-and-guide-lines-for-cad-packages/}{templates} to import these scripts to the EDA tool used by the designers (for this case the supported ones are KiCad, Altium and Eagle). It should be noted that DRC is usually checking for trace width, spacing and enclosure, but this is only a subset of the DFM. So it is quite important to have an overview of the methods that are going to be used for the fabrication of the PCB. DFM is sometimes referred not only to the bare board production but also to the assembly (Design for Assembly). Actually, DFM and DFA automatic checks are the first things that the manufacturers do. 

It is recommended to know in the design phase how your PCB will be assembled to integrate the corresponding considerations and rules. Some guidelines for DFM outside the scope of DRC could be:

\begin{itemize}
	\item All component outlines on your silkscreen should be marked with a reference designator and polarity (pin 1 marker) indicators. 
	
	About polarity, there is also the option to export a dedicated layer for this purpose from your EDA tool (Fab layer for KiCad) as we have already mentioned.
	% \href{https://www.altium.com/design-manufacturing-resources}{Altium source DFM}
	
	% \href{https://www.seeedstudio.com/blog/2019/06/12/how-to-generate-assembly-files-and-why-they-are-important/}{source for assembly data}
	
	%\item Placing and orienting components
	\item Prefer to place all the components to the top side of the board. Double sided PCBs are costly. If component placement is done with automatic pick and place machines, then additional cycle is required in the assembly line to flip the board and do the placement on the bottom side.
	\item Orient similar components in the same direction. It is easier for inspection and testing. It looks nicer too!		\item How my PCB is going to be assembled/soldered? Manually or by automated machinery? Reflow or wave soldering?
	
	If \textbf{wave soldering} is going to be used then for optimal soldering the designer should consider: 1) The SMDs are aligned perpendicular to the direction of the board going through the wave, 2) Large components should not "shadow" smaller ones and 3) Dual in line packages such as SOIC have their axis aligned to the wave direction. 
	
	%\href{https://www.vse.com/blog/2020/01/21/component-orientation-on-pcbs-best-practices-to-optimize-assembly/}{source}
	
	In general the assembly house could provide some guidelines analogous to the soldering method. 
	
	%Some packages may can't wave soldered properly like QFP, because leads are located in all sides and trailing pins should be shadowed.
	
	%Place the two pictures by the eagle autodesk. How to place two figures side by side? The bad and good
	
	\begin{figure}[h!]
		\centering
		\begin{minipage}[b]{0.4\textwidth}
			\includegraphics[width=\textwidth]{assets/good_wave.png}
			\caption{Good orientation for wave soldering,\href{https://www.autodesk.com/products/eagle/blog/top-10-pcb-component-placement-tips-pcb-beginner/}{eagle}} 
		\end{minipage}
		\hfill
		\begin{minipage}[b]{0.4\textwidth}
			\includegraphics[width=\textwidth]{assets/bad_wave.png}
			\caption{Bad orientation for wave soldering}
		\end{minipage}
	\end{figure}
	
	For manual assembly, consistency in the component placement can aid a lot the assemblers and make the process less error prone. For example it would be helpful if all ICs is oriented so the pin 1 is located in the same direction.
	% (source, the Orcad book in the DFM section)
	
	\item Although placing via on pads is great in high speed application, assemblers tend to recommend not doing that because solder can wick into the hole thus a poor joint is created.
	
	%\href{https://macrofab.com/blog/via-in-pad-pcb-design/}{source}
	
	\item Place 2 or 3 fiducial (reference points for the pick and place machine) marks, which should not be covered with the solder mask. Generally, these should be placed diagonally in the corners of the PCB. The pick and place machine finds these points using a camera and all components are placed on the coordinates relative to these points (tip: The bottom left corner of the PCB outline should be selected as the origin (0,0) point). 
	
	%\href{https://uk.beta-layout.com/pcb/technology/assembly_guide.html}{source}
	
\end{itemize}

It is worthy mentioning that some times \textbf{conflicts} emerge when trying to design for manufacturing and performance (functionality) at the same time. Some DFM rules may violate DFP ones and the opposite, that could increase the manufacturing cost. So a \textbf{trade-off} mindset is very common! It is also important to contact the assembly house to understand and identify the requirements of the DFM.
%For example double sided boards, or placing vias all over the place, vias on pads (some parts may not solder properly), routing topology and orientation.

Footprint design is also related with DFM. In order to meet manufacturing (soldering, stencil and soldermask) requirements and electric performance criteria, pad size for each layer (copper, stencil, soldermask) should be designed in a suitable manner. For this purpose footrpints are usually compliant with industry standards like IPC-7351.

\section{Testing}

The PCB development cycle is consisted by 4 main parts: 1) component selection and design layout, 2) manufacturing, 3) assembly\footnote{Fabrication and manufacturing sometimes can be referred to the bare board production and the assembly or only to the first one} and 4) testing. Testing is integrated both in the manufacturing and the assembly.

Each manufacturer for the bare board production is following certain guidelines to ensure and control the quality of the board. These are based on industrial standards. For example, Eurocircuits is compliant with the IPC-A-600 Class 2, the most used one in the industry. In summary, this manufacturer combine  automatic optical inspection, flying probe testing (doesn't need test fixture) and human inspection. \href{https://www.eurocircuits.com/making-a-pcb-pcb-manufacture-step-by-step/}{source}. A more detailed approach for the Eurocircuit's inspection workflow can be found here  \href{https://www.eurocircuits.com/blog/how-do-we-assure-the-quality-of-your-pcb-part-1/}{source the 3 parts blog}

Once the board is fabricated, the next step is the assembly process (PCBA). The assembly house is responsible to populate the bare board with components and to detect and fix the defects (mention defects like what?) along the way. Methods used to spot defects are typically:
\begin{itemize}
	\item Automatic optical inspection (AOI).
	\item X-Ray inspection, also called AXI.
	\item Flying probe testing. Probes moving around the board trying to contact the test points and components.
	\item JTAG Boundary Scan. Actually JTAG isn't used only for programming and debugging MCUs, but it was created initially for assembly testing.
	\item Functional testing. 
\end{itemize} 

\begin{figure}[h!]
	\centering
	\includegraphics[keepaspectratio, width = \textwidth]{assets/testing_overview.png}
	\caption{Testing overview by SMTA/TMAG TP-101E standard}
\end{figure}

But there is also a number of defects that can't be spotted by visual inspection (accessibility limitations), so a more complex testing method along with design requirements (test fixtures) is coming to the surface. 

This method is called in-circuit testing (\textbf{ICT} aka bed of nails) and can ensure that the board can move in the production line with zero defects. These defects include solder bridges, shorts, opens, resistance, capacitance, missed components etc. In short a group of probes are interfacing with usually one side of the PCB  by placing test points\footnote{Test point: A small exposed copper area used as a connection point to test circuitry on a PCBA} to all the nets. For this to happen, a test fixture is required that increases significantly the cost.

%\href{https://www.altium.com/solution/designing-for-testability-pcb-design}{source}

\begin{figure}[h!]
	\centering
	\includegraphics[keepaspectratio, width=\textwidth]{assets/bed_of_nails.png}
	\caption{A bed of nails test fixture \href{https://hackaday.com/2019/02/09/test-pcbs-on-a-bed-of-nails/}{source}}
\end{figure}

%E.g. it is also possible to boot the board to some bootloader and execute tests available in the bootloader, test Ethernet connections, test USB connections, ... No need to say that this comes at a cost.\href{http://www.nod-pcba.com/pcb-assembly-process/pcb-assembly-testing-methods-en.html}{source}
%For the last one, in circuit testing (is used also for simple defects like open, shorts etc.) and flying probes are the most known ways to address it. 
Assemblers charge the customer by the type of the testing service and by the hours needed which is a function of volume. In general testing is an expensive part of the cycle.
% (, typically the 30\% of the total cost. \href{https://www.altium.com/solution/designing-for-testability-pcb-design}{source about the cost}).
The cost for ICT is on average 20.000 dollars! 
%Another resource for the cost, \href{https://en.wikipedia.org/wiki/Flying_probe}{wiki} and \href{https://blog.matric.com/what-does-pcb-in-circuit-testing-cost}{source}. 
Testing for electric performance (verify proper operation for analog and digital circuits) could also be integrated in the provided services as part of the ICT by performing power up tests. It can test for functionality as well as assembly defects.

On the other hand, bed of nails technique has \textbf{limitations}. For example in packages like BGAs and in high density boards, placing test points is very inconvenient. So JTAG interface comes for the rescue and aids the assemblers for the testing process. Requirement for JTAG boundary scan is the device under test to have a built in controller or to be accessed implicitly by another device capable for JTAG interface (e.g. Testing a memory module via an MCU with built in JTAG).

% source for the jtag info is the eevblog guy, video what is jtag?

%\href{https://www.acceleratedassemblies.com/blog/in-circuit-testing-the-best-technique-to-detect-manufacturing-faults/}{source}

%The are several inspection methods, including hands-on inspection by a person, automatic optical inspection that relies on image recognition, and even x-ray inspection to look through components that may block an inspector’s view. \href{https://telancorp.com/pcb-assembly-process}{source}

%Πρέπει να σημειωθεί ότι δεν μπορούμε να καλύψουμε οτιδήποτε γύρω απο τον κόσμο του PCB design, προσπαθούμε κάπως να δώσουμε ένα ερέθισμα, μια τροφή για σκέψη...

Finally there is the \textbf{functional testing} (FCT). Its purpose is to simulate the environment in which a product is expected to operate (Does everything work together?). Functional testers typically use a computer that is connected to test points or a test-probe point in order to perform FCT.
%(For that purpose a range of different signals are applied to check the electric behavior of the board and to determine)
It can check if all external analogue and digital inputs and outputs meet the requirements and the specifications. Functional testing can use the JTAG or the test fixture for the interface. 

%\href{https://www.pcbway.com/pcb_prototype/PCB_Assembly_Functional_Testing.html}{source}

%source about the interface is the fact that test fixture act as a funnctional interface too and from the EEVBlog talking about JTAG

Does it \textbf{worth} to include test fixture and the additional \textbf{cost}? In circuit testing and flying probe testing is a quite common debate. In general for low volume, prototypes and low complexity boards, flying probe testing is the way to go along with visual inspection and functional testing. Flying probe testing is referred also as a type of in-circuit testing but without the need of the "bed of nails", the test fixture. The drawbacks are: more cycle time, less test coverage (how to test BGA with flying probes?), no power up test etc. But usually a combination of methods can also be used. 

% resource that combinations of methods can be used, \href{https://en.wikipedia.org/wiki/Flying_probe}{source}

A good resource about flying probe testing, \href{https://blog.matric.com/flying-probe-test-capabilities}{capabilities of flying probe testing}

%Along with most other tests, flying probe testing does not power up the circuit. So you don't get the true real-world look at your product the way an ICT gives you, \href{https://blog.matric.com/flying-probe-test-capabilities}{source}

%\href{https://www.vse.com/blog/2019/10/01/understanding-in-circuit-test-vs-flying-probe-for-your-pcba/}{source}

% more about ict vs flying, \href{https://blog.matric.com/ict-testing-vs-flying-probe-testing}{source}

%NICE: The advantages of ICT are that it can test for functionality as well as for assembly defects, \href{https://www.vse.com/blog/2019/10/01/understanding-in-circuit-test-vs-flying-probe-for-your-pcba/}{source}. I think the same applies for the JTAG

\subsection{Design for Testing}

%Each type of testing that the assembly house can provide, has some requirements that need to be addressed in the design stage. These can be for example the location and the type (size and shape) of the test points.

%As we have seen the list of the possible tests that an assembly house can provide, the design requirements for each kind of testing should be mentioned.

Among the first things that the designer should definitely ask is how the board is going to be tested (planning ahead is crucial). There are quite a few rules that need to be addressed for testing, especially for ICT and flying probe testing. Some of them could be:

\begin{itemize}
	\item Place test points to each net being accessible from the bottom or the top of the PCBA. Preferably, to minimize the cost, use only one side (no need for an extra cycle in the assembly line) 
	
	%\href{https://www.jjsmanufacturing.com/blog/9-pcb-assembly-design-guidelines-for-flying-probe-test}{source for flying probe testing}. For ICT it is strictly from the bottom? I think no
	
	\item Distribute test points evenly over PCB. Don't have too many test points in the same area. 
	\item Design the shape and size of test points according to the probes that are going to be used by the assembler.
\end{itemize}

A standard recommended for testability guidelines is the "SMTA/TMAG Testability Guidelines TP-101E".
%In large volume production testing (functional, electric performance) is integrated in the assembly stage. But testing for functionality and electric performance can be done also by the user. So the test points should be configured in a way to be compliant with the probes of the electric instruments that are going to be used to analyze the electric signals. The designer obviously should determine which signals should be analyzed and place the test points accordingly. It is worth mentioning that placing more than one test points for the power supply in different areas of the board is also a good trick to inspect the integrity of the power and the ground.

Functional and electric performance testing can also be made manually. In this case, similar with the above guidelines for the automatic testing, the designer should place the test points to the nets of interest being \textbf{accessible} from the top or the bottom. The electrical instruments that are going to be used should also be considered to adjust the \textbf{shape} of the test points with the probes. Prob tips can help to clip the test points without using hands to handle them. Sometimes for \textbf{practical issues}, using vias uncovered with solder mask, can make using electrical instruments more easier than just a flat pad surface. In high speed signals, getting an accurate view of the tested signal via contacting probes to test points can be quite challenging. Thus the electrical characteristics of the probes should be considered. For example, to minimize the inductance of a long ground lead, provide a ground point near the measurement of the signal to make the loop smaller for signal integrity. Finally, place test points everywhere, the more the merrier, but be mindful regarding the following statement. 

% source about minimize the loop for the probes, \href{https://www.eevblog.com/forum/eda/test-points-for-mediumhigh-speed-signals/}{source}

% Fun question: When probing, it is like stealing the signal from the circuit and isolate a specific track from the rest? I mean the signal isnt divided to circuit and oscilloscope? If I provide a low impedance path then the signal will go to the oscillloscope right? and not to the rest of the circuit? If I have a led for example, will it produce light, if I probe the signal with very low impedance?

%\textbf{Should I take into account the impedance matching with the probe and the signal of interest, test point?}. . To mitigate reflections and attenuation should the designer consider the impedance of the probe and of the test point?

% source for probe impedance and tet point:\href{http://www.sigcon.com/Pubs/news/3_2.htm}{this}

% manual testing, via vs pad, \href{https://electronics.stackexchange.com/questions/48557/testpoints-vias-versus-pads}{source}

% \href{https://starfishmedical.com/blog/pcb-design-tips/}{more info about test points for the manual part}

%You can add something like a life hack: Clipper for probe for the test point.

%A nice question for manually testing, \href{https://electronics.stackexchange.com/questions/308025/how-to-create-measurement-points-on-a-pcb-for-diagnostics/308032}{this}

% about probing tip, \href{https://electronics.stackexchange.com/questions/480846/what-probe-tip-to-use-to-clip-into-a-pcb-test-point}{source}


As we have already mentioned in the DFM, in a similar way the DFT can emerge conflicts with the design for performance (DFP). Creating holes to each net in a tight space isn't always an easy thing to do and in high speed applications, test points can cause performance issues. Some traces due to the above reasons might not be tested at all. Thus design for testing might have some drawbacks but it can save a lot of cost in the long run. 

%courtesy of EMA-EDA webinar DFT, what designer need to know about testing \href{https://pages.ema-eda.com/EMA-2020-Webinar-Desiging-for-Test}{source}

%\textbf{Resources} for the above is for sure the webinar EMA we have watched! And a lot of assembly house documentation and blog post. I dont think that you can find relevant literature resource for this. Only some kind og generic books like the Hitchikker guide and the orcad design book.



\section{Simulation and analysis}

Simulations can be thought as a method to predict the future that can save a lot of time and cost in the long run. Imagine to be able to know that your board is functional and meets the specifications only if it was manufactured. Then it would take probably ages to finalize the design. So the intermediate way to increase the confidence of the design is achieved by creating a virtual representation of the product's behavior. The virtual world is now our playground without worrying about the cost or the risk of failure. Basically, engineers are trying to catch up with potential issues that could eventually arise when the design becomes a real piece of hardware.

%By integrating them in the development cycle, engineers are trying to catch up with potential issues that could eventually arise when the design becomes a real piece of hardware. For practical reasons (less computational time and resources) it is recommended to simulate critical traces like clock or other high speed signals. 

Creating a virtual world that is identical with the physical one is quite a challenge. Thus for practical reasons (less computational time and resources) it is recommended to simulate only critical traces like clock or other high speed signals.

The simulation can be divided into three main groups: 

\begin{itemize}
	\item Simulate the \textbf{circuit model}, a translation of the board layout. 
	%(Consider the distributed model though. But schematic is a group of lumped elements related with each other.)
	Don't forget the distributed model and consider the interconnections as transmission lines.
	
	SPICE related tools for this type of modeling that are used in the industry are LTSpice, PSpice, Multisim, Proteus and the \textbf{QUCS} (open source). Also most of the EDA tools provide built in circuit analysis based on SPICE models. KiCad can support circuit simulation and is based on the \href{http://ngspice.sourceforge.net/ngspice-eeschema.html}{ngspice} engine.
	
	
	
	\item \textbf{Electromagnetic} analysis with \textbf{S-parameters} and a corresponding simulation tool. The S parameters can be obtained either using an electromagnetic simulator having as input a compatible version of the physical layout or can be measured directly using electrical instruments such as vector-network analyzer. 
	
	Some tools to simulate Maxwell's equations are: Ansoft’s High-Frequency Structure Simulator (HFSS), Clarity 3D Solver by Cadence, CST PCB STUDIO and some open source ones like \href{https://www.opensourceimaging.org/project/open-ems-a-free-and-open-electromagnetic-field-solver/}{openEMS} and \href{https://www.fastfieldsolvers.com/}{Fast Field Solvers}.
	%But when to use such tools? When you don't have uniform lines (discontinuities), like vias, packages, connectors and the designer wants to have an overview of the electromagnetic phenomena for these cases.
	Numerical solution tools like Matlab, Octave can be used for modeling with S-parameters too.
	
	\item Specialized \textbf{signal integrity simulators} like HyperLynx and Cadence.
	% A \href{https://www.signalintegrityjournal.com/articles/199-useful-sipi-tools-from-istvan-novak}{list} of useful tools. \href{http://www.sigcon.com/Pubs/straight/planningsi.htm}{source}
	
	These commercial tools have many benefits than the traditional circuit simulators. They can import a trace layout and perform a lot of automatic checks to find and solve problems about signal integrity. They use the IBIS model (we will mention it later).
\end{itemize}

\subsection{Circuit vs EM simulation tools}

Neither of these tools individually is enough to have a complete representation of the circuitry and to identify the SI, PI, EMC problems. EM field simulators can handle EMC problems, resonance and non uniform wave propagation and considerations regarding the trace geometry such as how bad is this slot in a specific return path. The circuit simulator can handle switching noise (ground bounce), near field crosstalk, transmission line propagation and reflections.	

PCB design is a very complex structure. Usually tools that solve 3D Maxwell equations can handle simple ones and require expertise and a very good understanding of electromagnetism. So the circuit simulation, if you need to choose between these two, in most cases is preferable. It is quicker and easier to use and can offer an adequate representation of the physical layout.

\subsection{Circuit modeling}

A very good \href{http://www.sigcon.com/Pubs/straight/planningsi.htm}{resource} by Howard Johnson for overview about simulation.

What is modeling? "Modeling refers to creating an electrical representation of a device or component that a simulator can interpret and use to predict voltage and current waveforms". The devices can be divided to the active (e.g. transistors) and the passive (e.g. interconnections)ones. For the first, there are two types of models, the SPICE and the IBIS.

The SPICE model is well known in the world of circuit simulation and is used in the analog simulation tools based on the SPICE engine. However, vendors usually struggling to offer these type of models for their products because precious information can be obtained regarding the design of the IC. Thus without revealing the intellectual property of the product, vendors usually provide the so called Input/output Buffer Information Specification (\textbf{IBIS}) model that run in special signal integrity simulators (or behavioral simulators) of the industry, like the ones mentioned before, while containing only the necessary data for this type of analysis. Simulations tools that can handle these models are mostly commercial and probably the most complete tool among them is the HyperLynx provided by Mentor.
%Can you view the IBIS model to contain valuable data about the rise time of signals though? Or something useful to determine how critical they are?
However, IBIS models can be useful, even without simulation, by \textbf{viewing} their data that contains among others the rise time of signals in the digital ports.

An extensive resource to IBIS models is \href{https://www.analog.com/media/en/technical-documentation/application-notes/AN-715.pdf}{this by Analog}. Nice wording too.

%It is worth mentioning that in digital circuits with complex devices like MCUs, there is no point creating a SPICE model for them. It is more convenient to simulate specific digital ports of interest. A simple simulation could include in the schematic capture a driver, a load and the interconnection, the transmission line with impedance set by the PCB layout. 
%(Its impedance is a function of the width and the thickness and can be calculated using online tools such as the Samacsys or integrated tools of the EDA tool)

These traditional \textbf{analog} tools, based on SPICE, can do the work for the common digital scenarios. The question is how to build or find spice circuit models for the transmitter and the receiver? How to approximate the circuit, how to test and model high speed signals? How to integrate the inductive and capacitance coupling to your circuit? Can 2D and 3D fields results interpreted as lumped elements and included in my simulation (because crosstalk can't be seen by a schematic, but affects the design)? Circuit modeling using SPICE models should be a topic for a new report!


\subsection{Coupling}

Capacitive coupling and inductive coupling. Crosstalk is actually undesired coupling. But coupling can use 

\subsubsection{Capacitance}
The first one is greater when two conductors are closed together, or the voltage difference is greater or the higher the dielectric constant, the greater the capacitive coupling. The longer the length, the more the coupling too. All of these stuff can be calculated with online tools. But when is too much crosstalk. Actually regarding the above by furthering apart unrelated signals and coming close the signal path with the return path this is efficient for reducing crosstalk, plus trying to minimize the coupling length when running in parallel unrelated signals, are the best things that you can do. 

Capacitive coupling (stray capacitance) is a function of the space between the conductors, the dielectric constant of the material between the conductors and the area of the conductors.	So by furthering the distance between unrelated signals and minimizing the loop of the signal and the return path, the capacitive coupling is decreased and increased correspondingly. Thus this technique mitigates in most cases the crosstalk regarding the capacitive coupling.

Having a high capacitance between ground plane and power plane is beneficial for power integrity, rail collapse in the PDN, because it would be harder for the planes to change the voltage potential, thus spikes and noise will not affect significant the goal to keep a clean voltage potential between them. So the bigger the copper area and smaller the distance between the planes, the higher the capacitance. Also the higher capacitance can act like a low impedance path between these planes and potential undesired high frequency current flowing in the power plane could return to ground without many issues (emission, distortion) due to the low impedance. Actually I am not very sure for the above. According to Eric Bogatin planes dont act very good as decoupling but as low-inductance path for the decoupling capacitors? What is this mean?

A lot of stuff overlap between, but just write and we will see.

So again. Is the capacitive coupling related with the capacitance. Of course, the greater the capacitance, the more the capacitive coupling.

Does the power and ground plane act as low impedance path for high frequency noise?
What we mean by low-inductance for the decoupling caps?

There are actually three types of capacitance that I am interested: capacitance between power and ground plane, that is according to the parallel thought model, area, space and distance. The second is signal and ground plane, and the same principles apply, the bigger the width and the smaller the distance and the bigger the dielectric constant the bigger the capacitance and the bigger the length, so greater capacitive coupling. The third is the stray capacitance, between adjacent conductors in the same layer, it is the coplanar capacitance actually. Actually by increasing the width the capacitance increases, lowering the distance the same and increasing the length the same. Be aware for the microstrip, the efective dielectric constant because you have dielectric of free space and FR4. Usually for all of these stuff, capacitance, inductance and coupling 2D field solvers can be used. For more accuracy, may 3D field solution is required.

When voltage change for magic reasons, there is displacement current, current travels through vaccuum due to the electric fields. Charge, field.

\subsubsection{Inductance}

%Amount of current, length of line, cross section if it increased number of rings decreased, magnetic field doesn't care about dielectric but ferromagnetic materials. Inductance is the ratio just like capacitance, charge and rings double but ratio stays the same. Inductance is how effective the conductor create magnetic rings, it is geometry based not current. For a given amount of current how much magnetic field can create? High inductance, can create strong magnetic fields.

%Mutual inductance the rings of the other contributing to the total number of rings. It is the amount of rings that circulate the other conductor but these circles generated by the other conductor.

%Voltage is induced to the change of the rings that surround the conductor. The noise created due to induced voltage to another conductor when we have change in the magnetic rings called crosstalk and it is usually undesired inductive coupling, this noise due to mutual inductance is called switching noise, because it happend when the current change either in magnitude or direction. Any change of the rings will result to induced voltage. Self inductance, mutual inductance, partial inductance (when we have a loop but we are isolating a segment of current). Current always running in a loop, so actually the self inductance is partial self inductance and mutual inductance is partial mutual inductance because there are also other current and we have isolated one conductor, that is why it is partial. Increasing the cross sectional area, the inductance decrease, increasing the length, the more magnetic rings per current. Less density, more inductance, the current should be distributed to lower the inductance


\begin{itemize}
	\item Inductance is a geometric dependent and indicates the efficiency of a conductor to create magnetic rings around it. It is a ratio.
	\item The more the rings, the bigger the inductance.
	\item Magnetic fields doesnt care about dielectrics but ferromagnetic materials.
	\item Inductance is a function of the cross area and the length. If the cross area is bigger, the current is distributed, the density is lower, the rings are less. The bigger the length, more rings I can count.
	\item When rings change induce voltage. How fast they change and how many rings affects the induced voltage
	\item Self inductance, mutual inductance, partial self, partial mutual, net aka total
	\item Mutual inductance is always less than the self
	\item The induced voltage due to mutual inductance is called \textbf{switching noise}, because it is created by changing the current that change the magnetic rings. Is this noise the cause of the return current? I mean didnt the return current created due to mutual inductance in the first place? I mean yes. In reality you may not have the exact current, because the current of the return will be such to opposite the change, and the change is produced by the mutual rings. Actually this induced voltage you will see later is the ground bounce, it is noise and it isn't the return current. In transmission lines we have seen that the return current is showing immediately, is this noise or the return current itself again?
	\item I am not sure to answer the above. Let's imagine a loop, current travels in loop, so we define explicitly that there is the same current and NOW we introduce the magnetic fields and the partial inductance. Voltage noise is induced also due to self inductance and this is the so called intertia.
	\item In the near field the inductive coupling is stronger. Dominates the h field to the e field. As we go further they match each other.
	\item When you have two parallel conductors having current running in the same and opposite direction, then the total inductance is increased by making it smaller and reduced by making it bigger correspondingly. So unrelated signals space them out and the signal and return path come closer. But why we want to make the inductance small, because inductance is the ability to create rings, less rings equals less (in intensity) radiation, poor induced voltage, poor inductive coupling, less potential for mutual inductance and crosstalk to other conductors nearby.
	\item Lets say we have a loop and current. The noise that will be induced in each conductor is defined by the change of the total rings. How many rings do I see in total taken into account the partial self and mutual inductance? In the opposite case the rings cancel and you have actually very little induced voltage. So it doesn't matter the length, but the distance of the loop. The length matters for the capacitive coupling. The induced noise voltage is probably referred to the self and the mutual, actually it doesn't matter because we see total not self or mutual. The total of the first and the total of the second that will be symmetrical too. 
	\item What is eventually ground bounce. It is the induced noise of the return path, when calculating the total number of rings surround it. The more the total rings, the change in them, will cause a greater induced voltage and ground bounce. The effective inductance in other words in each conductor will determine the induced voltage across the conductor. If the conductor is the return then the induced voltage is the \textbf{ground bounce}. The total numbers of rings surround each line will determine the amount of induced voltage, noise. We want igh mutual inductance in opposite currents.
	\item To minimize ground bounce, induced voltage in the return path you can 1) reduce the rate of change and 2) decrease the total inductance: 1) reduce the partial or 2) increase the mutual one. For the first one, short and wide traces, planes in other words close the space between signal and return
	\item By measuring the inductance, the partial, the mutual by spacing and geometry factors we can determince the total and thus the voltage drop. The rate of change in the current is the current divided by the rise time. So the voltage drop will be the total inductance multiplied by the rate of change. Rate of change, this is why rise time matters. When there is also a change in voltage, there is current among conductors spaced by dielectric. All of these stuff can be calculated. How much current? How much voltage? But let's stick to the inductance and the voltage drop.
	\item So the total inductance is a dance between the space and the width of the conductor itself, for the partial and the mutual. For different configurations you can estimate the total votlage drop.
	\item Things are worse when adjacent pins carrying power current like in most iCs. This is where current have the same direction. The mutual will be added to the self and votlage induced will be significant!
	
	\begin{figure}[h!]
		\centering
		\includegraphics[keepaspectratio, width = \textwidth]{assets/inductance_spacing.png}
		\caption{By eric bogatin p377}
	\end{figure}
	
	
	\item In general know what is the direction of current. Then space away the same and close the direction. The bigger the via, less the inductance.
	\item Loop inductance. Taking everything into account two partial self inductance and two partial mutual for the 2 wire conductor loop. So it is all about to minimize the area of signal and return. The bigger the area, the bigger the loop inductance as a whole.
	
	\item PDN, PI is signal integrity not for signals but for power and ground. The changing current when transistors are switching that passed through the impedance of the PDN interconnects will cause voltage drop, called rail rail collapse or rail drop. First goal keep the impedance low, to minimize any kind of voltage drop. Current still exist but no voltage drop significant. How much decoupling we need? Charge is flowing, all the charge will be used to charge the capacitor, the bigger the capacitance, more charge in the cap. Two planes from a capacitor, charges are stored. How much charge will flow away from the capacitor resulting in voltage drop? The amount of charge is the current of the chip. If the current is high, the capacitor will be decharged. The goal is to have a big capacitor to prevent any significant charge depletion due to current moving in the planes resulting to significant voltage drop. Two planes are not enough for the necessary decoupling capacitance and we need to have some more. Can much capacitance cause problem? Decoupling capacitors also used as low impedance for high noise. Rail collapsing noise. How much capacitance do I need?
	
	\item You could calculate how much current run through the chip by the power dissipation.
	
	\item Be aware of the parasitic elements of the interconnection of the passive devices. As frequency is getting higher the impedance of capacitor change due to the loop inductance. The rate of change becomes significant and the loop inductance can cause issue. So you need to be aware of the parasitic behavior of passive components and minimize the loop 
	inductance.	Actually as we have said everything has inductance and capacitance and it will get significant as the frequency increases, so check how much capacitance and inductance there is across the space using either 2D or 3D solver. In general by keeping traces short, wide and tight or further away, it depends on the current the case, the parasitic behavior will decrease. Simulate!!
	
	\item Everything has impedance and as the rate of change is getting faster, the parasitic behavior will be dominant! Parasitic we mean the inductance and the capacitance across the interconnections, capacitive coupling, inductive coupling, crosstalk and so on, are related with this.
	
	\item Holes in plane increase the inductance of the planes. Making the holes as small as possible is good for the plance but bad for the inductance of vias.
	
\end{itemize}



\section{DFP clear version. Actually not. Check above}

Now that we have seen the challenges of identifying transmission lines, it's time to lay down the problems and the solutions. As we have said transmission lines are associated with reflection, timing issues and characteristic impedance, there aren't electrical shorts. Actually isn't only transmission lines the problem. So we need to rephrase this.

Reflection can cause a lot of problems like ringing, . High frequency components are getting significant in terms of their magnitude along with emissions. 

How to structure this, bro? Are all the problems regarding the DFP wth transmission line. For DC what would be the problems? Actually isn't so simple, because let's assume that neglect transmission line phenomena, but you have misjudged the actually lenght of the interconnection, because there is a slot in the return path. So whenever you are in the zone of transmission line or not, you need to do the layout supposing you have high speed signals that can cause problems! So, we need to lay down problems and their solutions, transmission line or not. In the DC era, we care most for copper and current tolerance, thermal issues and what else? Not to have reflection and radiation by not knowing the return path! Plus you need protection for external noise that can transform your lines to high frequency components and so on. Okay we have stuff to do here. I will start with hitchhiker guide and will jump to eric bogatin. Again. Eric Bogatin is big but okay.

Can you have zero reflection but a lot radiation? I mean the length of the traces may be very short and this help with the wave nature, but what about the radiation? The radiation to happen I need accelerated charges going back and fourth. So even if I have short trace, if the charges are distributed to conductors and going back and forth I will have radiation? But this isn't transmission line effects. The radiation aspect, I can t actally understand how it correlates with the wave nature of interconnections. They may radiate, but if you don't have length of trace with the appropriate size the EM waves can't act like antennas to get the coupling. Coupling vs radiation?

coupling vs crosstalk vs radiation?
Crosstalk is undesired coupling, caused by near field radiation?
You may have but it is more significant when you have transmission lines. It is a function of how many wavelengths fit to the trace? Probably. Transmission lines doesnt not radiate because field cancellation, return path! What is radiation. It is EM waves, so you need a voltage difference and current. If the trace is ultra small, then there is practically no current no voltage drop, no EM fields, like DC right? So if you have a very high signal, and the trace is ultra small, it wont radiate or reflection. The goal is not to have voltage difference, so no current so no EM fields. The less the voltage difference, the less the signal travels to the interconnection, the less current, less EM fields, less radiation even if you didnt provide a good return path. So radiation is also connected with the rule of thumb of having or not having a transmission line. But what will happen when we will see a resistance as a load? There will be radiation, no matter the interconnection length right? Is this correct? No, because in DC we have also voltage drop but no antennas. Antennas exist only when the wavelength of the signal is relative to the length of the conductive piece. For accepting and radiating. If you have a very short trace with very short rise time, it will not radiate if it is ultra small. So okay the wavelength and the length of the conductor about radiation has a relationship...The question is why. I think I can somehow feel it. But even if trace is mall there is skinn effect, but radiation isnt related with voltage drop itself but relationship between wavelength and conductor length. If transmission line radiates then it isn't transmission line, this is why I cant index stuff with transmission line

THe higher the frequency, skin effect, resistance, current and EM waves. EM waves is a function of the voltage and the current. If there is current there are EM waves. Higher voltage due to ringing cause greater electric field and the same for the current and the magnetic field.

common mode noise, two loads sharing return path that has resistance, current in the return path will cause the lamp to shine!

\section{Design for Performance notes}

Somewhere I need to mention the path of the least impedance. There are I think some very good reference scattered around somewhere about this. All about circuits and my fav books...

Basically I think that I am gonna right a summary of the Eric Bogatin book from now on about SI, PI and EMI. I will check for some tips also the hitchhiker guide. Mark Montrose is also good and the transmission line book by IEEE. The Howard Johnson is maybe too much. Rcik Hartely maybe. But we are glas that Eric Bogatin writes well written stuff.



PCB layout placement. Make a checklist.

PCB routing guidelines for high speed applications.
\subsection{Wave coupling}
The impedance of free space is 377 ­, which makes it easy for the field to follow a new path between pairs of conductors. These new paths often lead directly into equipment. The field patterns are never simple and a WCC is needed to estimate the nature of the coupling. By Ralph page 101

All the problems of PCB design can be grouped to three main categories: Signal Integrity (SI), Power Integrity (PI) and Electromagnetic compatibility (EMC)

This section is by Ralph Morrison.

\subsection{Impedance matching}

The only thing that causes a reflection is a change in the instantaneous impedance the signal encounters. The instantaneous impedance the signal sees depends as much on the physical features of the signal trace as on the return path. (page 89)

What is the instantaneous impedance I see? Asking the signal when outputs from the driver. But what about the impedance of the circuitry inside the IC chip. What is the impedance from the transistor to the pad, is the actual question? I don't think that you should worry about that because the length is very small. You would be worried in very very high frequencies where everything is comparable to the wavelength of the signal, even the pads, the leads, the IC circuitry. So in our design we are interested in the impedance only when it matters, for the interconnections in other words. Ok this is great that we figure it out. Also the characteristic impedance is actually not a function of the length, but this is unrelated with the fact that we dont long transmission lines to enclose the fields and make the wave contribution smaller. Actually you need to search if the package taken into account controlled impedance interconnects. Actually I think that the designer cant do anything if there are reflections at the level of the pad. This is a bad IC design. Or it should be provided the impedance of the IC circuitry, or I think this isn't feasible. The key is to not have reflections in the pad, in the component. Then the layout designer should find the way to impedance match, or I think that I am not sure. 

What can cause a discontinuity? Whatever changes the impedance. A via, a gap in the return path, a change in the width of the track. But most of the problems arise by what is happening at the end of the net, the load? What is the impedance? The engineer what should do is to keep the impedance of the net constant at first. Then the engineer can worry about the driver and the load impedance.

Be aware the junctions when the signal is shared. The junction is a discontinuity and the signal travels distorted in the two lines. In this case when interconnection branches, consider rerouting the trace to be a daisy chain? Check more in the routing topology.

\subsection{Routing topology}

There are many ways to connect two points:

\begin{itemize}
	\item Point to point
	
The most common one.
	\item Daisy chain
	
Avoids the use of junctions to connect multiple pins together, one driver multiple loads and the avoidance of junction can help to see the signal a stable impedance along the way to minimize the reflections.

This method saves space and it is used for DDR memories
	\item Bus
	
Bus with stubs. Try to keep the stub as minimum as possible to avoid transmission line effects.
	\item Star
	\item Differential pairs?
\end{itemize}

About clock skew? What routing is the best? Tuning of course but which from the above categories?

\subsection{Signal integrity}

Signal Integrity: Involving the distortion of signals.

\subsection{Power integrity}

Power integrity: Involving the noise of the interconnects and any associated components delivering power to the active devices. 

\subsection{EMC}

Electromagnetic compatibility: The contribution to radiated emissions or susceptibility to electromagnetic interference from fields external to the product. When we discuss the solutions, we refer to EMC. When we discuss the problems we refer to electromagnetic interference (EMI)

\section{SI, PI, EMC}

There are 6 main categories that every problem related with SI, PI, EMI falls of: by Eric Bogatin and then investigate each one of. Actually I think that I should made a summary, overview of Eric Bogatin. I mean it contains basically everything.

\begin{itemize}
	\item Signal integrity effects on one net
	There are three generic problems associated wit this. Reflection, dielectric losses, timing issue. Dielectric losses is for the Gigahertz region!
	
	Impedance matching falls of to the reflection aspect. The ringing is the outcome of impedance mismatch sometimes. Why reflection is bad though in the first place? When reflection occurs, a portion of the signal is reflected back, then the signal is following its way distorted of the driver signal. The reflection can also trigger the logic gates too and this is quite bad. About impedance matching we have a whole section too.
	
	
	Timing. Timing issues are called skew. False triggering. Tuning the traces, matching the length. When we need to consider time constraints though? The simulation will show you the answer maybe? Actually hor high speed applicaton, tuning and impedance matching is two things that the designer should think about, \href{https://www.altium.com/documentation/altium-designer/length-tuning-ad}{source}. So if it is a transmission line, then we should not only match it but tune it too? but transmission line vs high speed signal? Tuning in differential signaling most. In cases where I should receive tow signals in the same relative time. It is all about relative things too. Okay I think I got it.
	
	\item Crosstalk
	
	Capacitive and inductive coupling. What is far and near end?
	
	The inductive coupling, mutual inductance dominates when the return path isn't a wide plane and then ground bounce can happen. This happen especially to connectors and packages. Trace geometry should be considered to mitigate the inductive coupling, the mutual inductance and the ground bounce as the effect of these.
	
	\begin{figure}[h!]
		\centering
		\includegraphics[keepaspectratio, width = \textwidth]{assets/field_cancellation.png}	
		\caption{Field cancellation}, \href{http://www.signalintegrity.com/Pubs/edn/FieldCancellation.htm}{source}
	\end{figure}
		
	How to address crosstalk: Smaller rise time, lower dielectric constant, short lines aka interconnections, minimize the loop area, further the distance between them, wide reference plane unbroken

	\item Rail collapse noise. This falls of to the PDN. Most of the problem originate from poor ground and power.

	It is crucial to provide low impedance paths for power and ground low. Power supply is one of the fundamental blocks of the design because coupled noise can affect the whole circuitry. If these power lines have significant impedance, then when current changes voltage drop will appear. This can result to ground that isn't ground and power that isn't power. The voltage difference due to the voltage drop can cause logic failures, false triggering, bit errors, jitter etc. 
	
	If something has low votlage but high consumption then this implies that there are high currents and then impedance and voltage drop becomes much more crucial because the voltage noise tolerance is less and voltage drop increased due to the high current. This is why low impedance paths, such as planes, large areas of copper is suitable as a solution to this problem.
	
	\textbf{More delicated version}. Everything has resistance and when frequency gets higher everything has reactance too. Out goal for the PDN is to provide low impedance paths to avoid significant voltage drops that can lead to false triggering and circuit failures. Especially for applications with low voltages and high power consumption, it is more crucial because high currents will contribute to significant voltage drops and the ground won't be ground eventually. (Remember that ground also serves as the return path of the signal, there is current in your plane! It isn't just for reference). So it is not about frequency, is about magnitude. But if you have high current with high magnitude?
	
	The path of the least impedance
	
	\textbf{A word about planes}. Spanning the large ground plane under every component may not even be desirable in high-frequency applications. For example, in high-frequency mixed-signal circuits driven by crystal oscillators, placing a ground plane directly below the signal clock creates a center-fed patch antenna. This will actually exacerbate EMI issues, and signal integrity is likely to be degraded without significant shielding. by \href{https://resources.altium.com/p/preventing-ground-loops-your-pcb-design}{Altium}. 
	
	What about the filtering aspect?
	
	Things you can do are: 1) closely tight adjacent planes, to create a big capacitor that cant charge or decharge easily, so it has a "constant" voltage difference. 2) Decoupling capacitors, this is for the filtering right? Why noise is high frequency though? Prefer packages SMDa and with no leads, short leads actually. (short and wide is the best). Ceramic capacitors behave better with high frequency. Dont forget the parasitic nature of these two. Usually in each IC you could place a large and a small one, to smooth the voltage spikes and filter high frequency noise?, \href{http://www.capacitorguide.com/coupling-and-decoupling/}{source1 for decoupl caps}. Basically you have two caps to smoothen different types of noise, low and high, \href{https://www.autodesk.com/products/eagle/blog/what-are-decoupling-capacitors/}{eagle source}. Capacitors act like a very low impedance path for AC signals. Thus the noise will be redirected to a low impedance path to the ground plane. But where this noise will go the ground plane? To its source and what if the return path will overlap with other return paths? Can currents interfere? If I have a high speed signal, I know that the return will be exactly under the signal trace. But if I have very low ones, that will follow the shorter distance? And you don't know the return path of noise, so how do you know where the noise will return? Can the noise dissipate in the ground plane? Actually capacitors store energy, the noise is stored? and it released only if the voltage is under the charged voltage. Actually the thought that ac is passed through capacitor is bad. Capacitors in order to charge they need time, if the voltage going to change instantly then the necessary build up charge will not take place in the cap to change its voltage. A property of the capacitor is that it takes time to change it voltage. Actually there is a return current when a cap is charged. The return path will be under the trace that picked the noise actually, yes this is it. The traces doenst know who is the source, they only deliver what they see, if noise is coupled then I will deliver it too. Τι ωραία που τα λες ρε μπαγάσα.
	
	\begin{figure}[h!]
		\centering
		\includegraphics[keepaspectratio, width = \textwidth]{assets/decoupling.png}
		\caption{\href{http://www.learningaboutelectronics.com/Articles/Decoupling-capacitor.php}{source}}
	\end{figure}
	
	\item Electromagnetic interference. Two things, emission and susceptibility
	
In order to have an EMI problem you need three things: a source of noise, a medium (pathway) and an antenna that radiates the noise. Sometimes the noise on the board will not cause any issues in terms of SI and PI, but EMC tests can still fail. It is very important for a product to pass EMC tests, otherwise harmonics will interfere with other communication bands violating the allocation of the frequency spectrum (I think this is the primary reason. It is a standardized process to be legal in the job market I guess). Usually techniques used to address noise about signal integrity and power contribute to EMC, but there are also some exceptions. What should be your primary concern? Do you pass EMC or not? What are these tests actually?

Shielding is a common technique (Faraday cage) to reduce emissions. Regarding EMI problems, engineers should be very careful with the cables and input output ports. Typically they can act like antennas and radiate. That's why ferrite bead, chokes are used, especially for twisted pair ones. 
%Be careful with the cables and input output ports that is used for interfacing. These are most of the times where the emission is take place or where the product is susceptible.
	
\end{itemize}


\section{Checklist}

\begin{itemize}
	\item Did you read the datasheets and the docs in general for the hardware development of each component offered by the vendors?
	\item Did you place decoupling capacitors?
	\item Did you know the logic family, the voltages, the current, the rise time? How critical is your design?
	\item Do I need to take into account reflections? 
	\item RF, Digital, Analog?
	\item Are you sure that everything fits? Mechanical constraints?
	\item Did you run the DRC check?
	\item What are the tolerances, the noise margins? 
	\item What was the trade offs between cost and functionality to meet the requirements?
	\item Did you design for manufacturing and testing? Who is gonna do the testing?
	\item Did you make simulation to increase the confidence for the functionality of your board? 
	\item Are the assembly and manufacturing data with the proper format and compliant with the needs of the corresponding house?
	\item Are you sure that ground is ground?
	\item Are passive components like capacitors and inductors what they are supposed to be? Parasitic elements?
	\item Do you like the aesthetic aspect of your PCB (PCB is art!)? If yes then you are probably going well.
	\item Do you plan to improve this checklist? (It would be wise to do so!)
\end{itemize}

\section{Rules of thumb}

\begin{itemize}
	\item Know your design, the rise time, the critically of the signals, the tolerance, the noise margins.
	\item Keep current loops small.
	\item Avoid ground loops. Tie them to the ground the fastest way.
	\item If you can do your job with higher rise time then do it.
	\item Low inductance everywhere: 1) Wider and shorter the better, 2) signal and return as close as possible, 3) Unrelated signals as distant as possible.
	\item Identify the return current paths. Be aware of slots in the ground plane and the holes by vias.
	\item I/O connectors to the edge of the board, isolate them.
	\item Partition your design by functional groups, RF, analog, digital, high-speed
	\item High speed signals when changing layers wants to return too. Don't forget to provide a low impedance path for the return using a capacitor. How to pick this capacitor? What is the range of the frequencies of your signal that you want to see the capacitor as low impedance? 
	\item Use circuit models and 2D, 3D fields to translate the geometry characteristics to electrical elements to understand the performance of your board.
	\item Daisy chain for controlled impedance is preferable. When impedance matching avoid junctions, branches and stubs. Don't forger the termination where it is needed.
	\item PDN, accumulation of currents, high current, be careful with thermal performance and if the trace can handle it. How much current can my trace handle? How much trace width without overheating and damaging the board? Do I have high currents? Does my trace width can handle it, without overheating and damaging the board?
	\item Design is just a portion of the PCB making. It should be fabricated. Know the process and the requirements and integrate them to your design (DFM, DFT)
	\item Avoid stubs, either formed by traces or vias.
\end{itemize}

\section{Summary}

Along the way we have mentioned a lot of rules of thumbs but we didn't mention the most important one. Don't take rule of thumbs too seriously. These are approximations and sometimes can differ a lot from reality. If designer want to think something fast then it is an okey method. But simulations along with specific data for your application is a more accurate but time consuming process. 

\section{Endnote}

Warning, a biased opinion is following: If Balanis book is the bible for Antenna theory then Signal Integrity and Power Integrity by Eric Bogatin is the bible for PCB design.

\section{Resources}

Except some individual technical articles that we found along the way, there are some very basic resources:
\begin{itemize}
		\item Eric bogatin book for signal integrity
		\item Ralph morisson, field and electronics
		\item Altium live, Daniel Bereek
		\item Some application notes that are scattered all around the globe, nxp, freeescale, ti, maxim integrated
		\item Howard johnson, black magic
		\item Mark montrose about EMI
		\item the orcad book
		\item Rick hartley
		\item Douglas brooks
		\item edn 
		\item ee times
		\item all about circuit technical articles
		\item electronic design
		\item altium live
		\item fedevel academy
		\item altium youtube channel
		\item papers
		\item Orcad book
		\item learnemi. The best.. Electromagnetic radiation
		\item blogs of eagle/autodesk, orcad, altium
		\item Transmission line analog and digital
		\item antenna-theory
		\item Stack exchange
		\item Each software vendor, will probably have a list of educational videos in their websites. For example ema-eda, altium, pads and so on. It is worth checking them out. You can find a lot about simulation, a lacking thing of our designs.
		\item I think \href{https://www.technolution.eu/uploads/2013/08/objective-white-paper-nr5-2006.pdf}{this} is awesome. This is called as white paper.
		\item blog matric, pcbway, vse!
		\item Another amazing book by analog devices. PCB design issues and simulation, \href{https://www.analog.com/en/education/education-library/linear-circuit-design-handbook.html}{this}
		\item \href{https://www.signalintegrityjournal.com/}{signal integrity journal}
		\item \href{http://www.sigcon.com/}{signal integrity Howard Johnson}
		\item A very nice video about pcb layout by phill for KiCad, \href{https://www.youtube.com/watch?v=t5phi3nT8OU&t=4940s}{this}
		\item also this by eevblog for pcb review, \href{https://www.youtube.com/watch?v=xhRhsCVF8mE}{this}
\end{itemize}

\subsection{Books}
\begin{itemize}
	\item Transmission Lines in Digital and Analog Electronic Systems: Signal Integrity and Crosstalk
	\item Signal and Power integrity demystified by Eric Bogatin
	\item Complete PCB design using OrCAD
	\item Mark Montrose EMC Compliance
	\item Right the first time by Rick Hartley
	\item Ralh Morisson EM fields in elecrtonics
	\item High speed design: The black magic by Howard Jonhson
\end{itemize}

\subsection{Notorious websites}
\begin{itemize}
	\item \href{https://bethesignal.com/bogatin/}{Be the signal}
	\item \href{http://www.sigcon.com/}{Howard johnson}
	\item \href{https://www.signalintegrityjournal.com/}{Singal integrity journal}
	\item \href{https://www.edn.com/}{edn}
	\item \href{https://www.allaboutcircuits.com/}{all about circuits}
	\item \href{https://incompliancemag.com/topics/resources/}{incompliance magazine}
	\item \href{https://electronics.stackexchange.com/questions/15135/decoupling-caps-pcb-layout/15143#15143}{An amazing stackexchange answer. Definitely worthy to give it a read!}
	\item \href{http://www.hottconsultants.com/}{Henry Ott}
	\item \href{https://learnemc.com/}{LearnEMC}
\end{itemize}

\subsection{See also}

\begin{itemize}
	\item \href{https://drive.google.com/file/d/1C4cFAzJpTlKedcgKvlWDQnTaCCgdo58s/view?usp=sharing}{EMC Fastpass. Getting EMC design right at the first time}
	\item \href{https://drive.google.com/file/d/1mr8UNMDeXmdCBnOVq22b1YUi3WwKNUMm/view?usp=sharing}{Texas Instruments. High-Speed Layout Guidelines}
	\item \href{https://drive.google.com/file/d/1gwrVG8WULKCOxYYrVLvCCZh1-luvQacq/view?usp=sharing}{NXP. High-Speed Layout Guidelines}
	\item \href{https://drive.google.com/file/d/1ylptbGbczsr2scbjPCba7q4J1PiyvQL8/view?usp=sharing}{David L. Jones. PCB Design Tutorial}
	
\end{itemize}

\section{Nice random things}

\begin{itemize}
	\item Soldering is used both to attach components physically to the PCB and to provide electrical conductivity between the component’s leads and the PCB traces. By the orcad book in the chapter of DFM
	\item As the frequency goes up the capacitor effect dominates the impedance and can filter out high frequency components. So a probe with high capacitance due to load sharing will effect the measurements of the board. How an oscilloscope works? Check \href{https://www.allaboutcircuits.com/worksheets/basic-oscilloscope-operation/}{this}
	\item High speed design is considered when clock frequencies are above 100 Mhz or rise time is under 1 nano second by Eric Bogatin SI and PI simplified page 82. It is all about how transparent are the interconnections and if they are actually electrical shorts!!
	\item The only book that you need to read to design a PCB is the Eric Bogatin stuff
	\item Interconnection ways, routing topology, \href{https://resources.pcb.cadence.com/blog/pcb-routing-topologies-demystified}{source}
	\item In the same way you can have electric field with zero charge, imagine electric dipole in the far field, you can have current with no magnetic field, field cancellation signal and return path, \href{http://www.sigcon.com/Pubs/edn/FieldCancellation.htm}{link}
	\item Inductance, store of magnetic energy, capacitance store electrical energy. If we have strong em field we have high capacitance and inductance and coupling can occur. If these fields are supprresed like the signal and the return the total EM field is almost zero, very little inductance and capacitance then, right, yes. Trying to connect the abstract circuit theory with electromagnetism. 
	\item return currents should not overlap
	\item Signals are propagating electromagnetic waves.
	\item What is the \textbf{eye} diagram?, \href{https://www.testandmeasurementtips.com/basics-eye-diagrams/}{source}. A very nice and quick visualization method to review the signal integrity of digital signals regarding the usage of oscilloscope for measurements.
	\item A complete engineer should have probing skills and general to be able to handle electrical instruments, simulations kills, a solid understanding of field theory, transmission line and circuit theory and to be able to connect all of these dots regarding PCB design.
	
	\item place a capacitor in a slot, \href{https://ieeexplore.ieee.org/document/801393}{source}
	
	\item List of transmission line effects, \href{https://www.pcbway.com/blog/Engineering_Technical/PCB_circuit_design_transmission_line_effects.html}{source}
	\item Very good overview of EMC for PCB, \href{https://electronics.stackexchange.com/questions/81761/whats-radiating-on-my-pcb}{this}
	\item How to simulate a capacitor parasitic elements, \href{https://incompliancemag.com/article/impact-of-a-trace-length-on-capacitor-frequency-response/}{source}
	\item High speed design guidelines by nxp. \href{https://www.nxp.com/docs/en/application-note/AN2536.pdf}{this}
	\item \textbf{STACK EXCHANGE IS AWESOME}, \href{https://electronics.stackexchange.com/questions/15135/decoupling-caps-pcb-layout/15143#15143}{this}
	\item for pdn, nice example, article \href{https://benthamopen.com/contents/pdf/TOOPTSJ/TOOPTSJ-5-51.pdf}{source}
\end{itemize}

\section{Things to discuss}

\textbf{Brainstorming}: DRC can also be used for assembly rules like the size of the pads for reflow soldering? Most of the times, the footprints should be designed in such a manner that will take into account the assebly process. There many online footprint datasets that you can download a footprint, but it would be wise those footprints to be compliant with standards. Those footprints have probably the neccessery metadata for the reflow soldering. But orientation that is discussed below for the wave soldering is something that the designer must do. But can DRC do this? \textbf{How the DRC, DFM and footprints and reflow soldering are linked together?}

Actually regarding the above, the footprints should be designed in such a manner to meet soldering requirements and assembly requirements in general. For this purpose the footprints should be compliant with standards. Prisma in an email said something about it. Altium footprints, online footrpints, default libraries in EDA tools and so, it is something we need to taken into account. But it isn't listed in the design for manufacturing right? Yes but we should mention the importance of footprints compliant with standards.

To match the impedance, you need to know the impedance of the source and the impedance of the load. How to find them these two in the datasheets? The designer is involved with the interconnect. Now it is time to talk about 50 ohm standard. In rf application, IC are designed in such a way having the impednace matching in their strategy, so the IC design has 50 ohm impedance. Right? But from where to where? (\href{https://www.allaboutcircuits.com/textbook/radio-frequency-analysis-design/real-life-rf-signals/the-50-question-impedance-matching-in-rf-design/}{source})

How to find the output impedance of a pin. Datasheet!!, \href{https://electronics.stackexchange.com/questions/127046/what-is-output-impedance-of-a-pin/127050}{source}

When capacitors act like low impedance path? How it relates with self resonance, is it because the ohm resistance that the current flow? It is because of the charging, decharging that current can flow, but how it relates the fequency with the change of impedance with ohms and such.

\begin{itemize}
	\item How to find the input and output impedance of an IC chip?
	\item How to design a transmission line, microstip or stripline of the impedance matching. Do I need passive components or only the geometry of the track?
	\item How to find the rise time and the logic level of the chips that are used in my design? First of all, are they compatible? How do you make the component selection at first place? How to choose logic family?
	\item There are so many rules of thumb on when to worry about transmission line phenomena. But how can you actually make a simulation to see how the reflections for the frequency of interest contribute to the functionality of the board? How to define the noise margins of your devices? The circuit tolerance?
\end{itemize}

What is the source of the rate of change, what is the source of the transmission line? The power supply or the switch?

\section{Ideas}

\begin{itemize}
		\item Place the quote of the altium live, Daniel Beerek first!
		\item TO find a summary resource check the review section of the ralph morisson
		\item Replace the PCB design guidelines with DFP, design for performance, electric performance.
		\item the design checklist. What should be the flow of the design? What thought experiments I should make with my brain?
\end{itemize}

\section{Garbage collector}

\subsection{Production development cycle}

\subsubsection{Brainstorming}
What is the hardware design flow? Design layout, actually requirements, design concepts, problem statement, what this pcb we want to do, component selection, design layout, manufacturing, assembling and testing integrated in the last two steps, functional testing, is my pcb working as expected, electric performance, verification, validation, ready! Reference the webinar from EMA, orcad, design for testing!

Turn a concept, an idea to a product

First of all you should ask, who is going to do the test? What is the deal with the assembly house and what test are part of the cycle, flow? ICT solution is quite costly? Is prisma offering this service to our contract, sponsorship? 

Functional test is the last step of the fabrication, where we are testing if the PCB is operating, functioning as is supposed to do. Demo runs, demo environments. In space would be vacuum tests and other things too. A larger cycle than the common industrial one.

Εσυ μπορείς να συμγωνήσεις αν ο assembly house θα κάνει το testing. In large volumes you cant do it alone, but στην δικιά μας περίπτωση in circuit testing may be overkill. We can do some sort of in circuit testing using an oscilloscope. Πάντως συνηθίζεται το testing να είναι κομμάτι του pcb assembly, but what kind of testing, which kind of services? The assembler charrges you by service and hour. What type of testing do you want and for how many boards? \href{https://www.youtube.com/watch?v=UESc7ms4efo}{source}

\subsubsection{Testing}
Λοιπόν, όσον αφορά το testing, υπάρχουν πολλά στάδια. Πρώτα έχουμε quality control, testing in manufacturing level, bare board, then assembly testing και έπειτα functional testing. Ορισμοί για τα παραπάνω και τι ακριβώς σημαίνουν;

Manufacturing defects.

An example of an industrial production development cycle is the following:

\begin{figure}[h!]
	\centering
	\includegraphics[keepaspectratio, width = \textwidth]{assets/production_cycle.png}
\end{figure}

\end{document}
